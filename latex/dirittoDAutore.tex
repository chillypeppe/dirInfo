\newpage
\section{Diritto d'autore: tutela giuridica delle opere dell'ingegno}

\subsection{Che cosa tutela la legge sul diritto d'autore? Esempi}
Facendo fede al Diritto d'autore riferito nel \textit{Libro Quinto} del codice civile,
formano oggetto del diritto di autore le opere dell'ingegno di carattere
creativo che appartengono alle scienze, alla letteratura, alla musica, alle
arti figurative, all'architettura, al teatro e alla cinematografia, qualunque
sia il modo o la forma di espressione.\newline
Dietro all'atto creativo c'è un impegno, un lavoro e un investimento di
tempo (spesso anche di denaro), che rendono quindi l'atto creativo un'attività
da tutelare riconoscendo i meriti del suo creatore. L'oggetto di tutela del diritto
d'autore può essere determinato da una teoria elaborata dal giurista
Josef Kohler all'inizio del ventesimo secolo.\newline
Secondo questa teoria si fa una distinzione tra forma esterna, forma interna
e contenuto dell'opera creativa. Per \textbf{forma esterna} si intende la forma
originaria nel modo in cui viene espressa per la prima volta e questa
forma viene completamente tutelata dal diritto d'autore. Per \textbf{forma interna}
invece si intende la struttura espositiva (per esempio la struttura narrativa o
personaggi in un libro). Infine il contenuto di un'opera è l'argomento che viene
trattato, le idee, i fatti, le informazioni, le teorie in quanto tali,
a prescindere dal modo in cui vengono elaborate o esposte. Quest'ultimo
non è tutelato dalla legge sul Diritto d'autore. \newline
Definiamo \textbf{Creatività} come l'apporto personale dell'autore (minimo
grado di originalità e novità rispetto alle opere preesistenti) e \textbf{Originalità}
come il discostarsi da qualunque altra opera (evitando il plagio). Dunque
la legge sul Diritto d'autore tutela (22 aprile 1941, n.633) :
\begin{itemize}
    \item Le opere dell'ingegno creative (artistiche, letterarie, ingegneristiche ecc...)
    \item I programmi per elaboratore
    \item Le banche dati, intese come raccolte di opere, dati o altri elementi
    indipendenti sistematicamente o metodicamente disposti ed individualmente
    accessibili mediante mezzi elettronici o in altro modo. La tutela delle banche
    di dati non si estende al loro contenuto e lascia impregiudicati diritti esistenti
    su tale contenuto
\end{itemize}
Alcuni esempi sono:
\begin{itemize}
    \item Opere letterarie, drammatiche, scientifiche, didattiche, religiose,
    tanto se in forma scritta quanto se orale;
    \item I disegni e le opere dell'architettura;
    \item \textbf{I programmi per elaboratore}, in qualsiasi forma espressi
    purchè originali quale risultato di creazione intellettuale dell'autore.
    \emph{Restano esclusi dalla tutela accordata dalla legge le idee e i principi
    che stanno alla base di qualsiasi elemento di un programma}, compresi quelli
    alla base delle sue interfacce. Il termine programma comprende anche
    il materiale preparatorio per la progettazione del programma stesso;
\end{itemize}
Si pone dunque una particolare attenzione al fatto che ciò che viene tutelata
è la \textbf{forma}, \emph{non le idee o il contenuto}. \newline

\subsection{Chi è titolare del diritto d'autore?}
In genere l'autore, in alcuni casi diversi (es. datore di lavoro per il software del dipendente,
pubbliche amministrazioni). \newline
Le amministrazioni dello Stato, delle regioni, dei province, dei comuni sono
titolari del diritto d'autore sulle opere create e pubblicate per loro conto
e a loro spese.\newline
Lo stesso diritto spetta agli enti privati \textit{non-profit}, alle accademie, agli altri
enti pubblici culturali, salvo diverso accordo con gli autori. \newline
Il sistema del diritto d'autore è strutturato in modo tale da garantire
una quota o una parte della proprietà intellettuale su una data opera a coloro
che contribuiscono in maniera significativa al processo creativo, e da garantire
loro di essere considerati coautori, soprattutto nel caso in cui il contributo
di ciascuno alla creazione finale non possa essere distinto.\newline
Nella realizzazione di un'opera collettiva essa è sottoposta a diritti d'autore
sulle singole parti relativamente ai propri autori e al diritto sull'opera collettiva
che spetta a chi la organizza e la dirige (es. enciclopedia, antologia). \newline
Si applicano le norme sulla comunione previste nell'art. 1100 del Codice Civile: \newline
\emph{
    Quando la proprietà o altro diritto reale spetta in comune a più persone,
    se il titolo o la legge non dispone diversamente, si applicano le norme seguenti
    \begin{itemize}
        \item Le quote dei partecipanti alla comunione si presumono uguali
        \item Il concorso dei partecipanti, tanto nei vantaggi quanto nei pesi della comunione,
        è in proporzione alle rispettive quote.
    \end{itemize}
}

\subsection{Quando sorge il Diritto d'autore?}
L'autore acquista il diritto sull'opera al momento della sua \textbf{creazione}. \newline
Per essere tutelata, l'opera deve essere realizzata, concretizzata, esteriorizzata. Essa è tutelata
qualunque siano la forma e il modo di espressione. (NB. Non è tutelata una semplice idea).

\subsection{Quanto dura il Diritto d'autore?}
I diritti riconosciuti in capo all'autore di un'opera si distinguono in diritti morali e patrimoniali. \newline
Si dice che il diritto d'autore è una forma di diritto la cui durata di tempo è
\emph{illimitata}, cioè continua a protrarsi anche dopo la morte dell'autore stesso. Questa affermazione è da specificare
a seconda di quale diritto d'autore si tratta:
\begin{itemize}
    \item Diritti morali: sono diritti esclusivi riconosciuti dalla legge a tutela della
    personalità dell'autore. Possono essere esercitati se una loro violazione possa recare pregiudizio
    all'onore e alla reputazione dell'autore. Godono di alcune proprietà:
    \begin{itemize}
        \item Sono imprescindibili; \textbf{i diritti morali d'autore non scadono mai}
        \item Sono indipendenti dai diritti patrimoniali; cioè i diritti morali
        spettano all'autore anche se i diritti patrimoniali sono stati ceduti
        \item Sono irrinunciabili; cioè l'autore non può rinunciare ai diritti morali
        \item Sono inalienabili; l'autore non può cedere o vendere i diritti morali d'autore
        \item Attribuiscono il diritto di rivendicare la paternità dell'opera
        \item Attribuiscono il diritto all'integrità dell'opera
        \item Attribuiscono il diritto di \textit{pentimento}
    \end{itemize}
    Dopo la morte dell'autore possono essere esercitati da coniuge, ascendenti e discendenti diretti
    (anche non eredi).\newline
    NB. In caso di finalità pubbliche i diritti morali d'autore possono essere esercitati
    anche dal Presidente del Consiglio dei Ministri
    \item Diritti patrimoniali: il titolare ha il diritto esclusivo di utilizzazione economica
    dell'opera nel suo insieme e in ciascuna delle sue parti (es. mediante pubblicazione, riproduzione ecc.). \newline
    \textbf{I diritti patrimoniali d'autore durano per 70 anni dopo la morte dell'autore}.\newline
    Nel caso di diritti patrimoniali delle \textbf{Amministrazioni dello stato e amministrazioni
    locali, i diritti d'autore patrimoniali durano 20 anni dalla prima pubblicazione}.
\end{itemize}

\subsection{Quali sono i diritti morali d'autore?}
I diritti morali d'autore sono:
\begin{itemize}
    \item Il diritto di rivendicare la \textbf{paternità} dell'opera: \newline
    diritto di essere indicato e riconosciuto pubblicamente come creatore dell'opera.\newline
    Comprende anche il diritto di far circolare la propria opera in forma anonima o
    sotto pseudonimo. \newline
    L'autore può pretendere che il proprio nome sia indicato sugli esemplari dell'opera
    o in occasione di qualsiasi forma di utilizzazione o comunicazione al pubblico
    (es. esecuzione, rappresentazione, proiezione cinematografica, diffusione ecc.)
    \item Il diritto all'\textbf{integrità} dell'opera: l'autore ha il diritto di opporsi a qualsiasi
    deformazione, mutilazione od altra modificazione, ed a ogni atto a danno dell'opera stessa, che possano
    essere di pregiudizio al suo onore o alla sua reputazione.\newline
    Il diritto può riguardare modifiche dell'opera e anche modalità di comunicazione
    che possano cambiare la percezione e il giudizio del pubblico (es. utilizzo in pubblicità).
    Questo diritto non può essere utilizzato se l'autore è a conoscenza delle modifiche e le ha accettate
    \item Il diritto di \textbf{inedito}: l'autore ha l'esclusivo diritto di decidere se e quando
    pubblicare la sua opera. Egli infatti, se vuole, può lasciarla per sempre inedita (non pubblicarla
    mai). Inoltre se l'autore ha espressamente vietato la pubblicazione, i suoi eredi dopo la sua morte
    non potranno pubblicare l'opera inedita.\newline
    L'autore può opporsi alla prima pubblicazione e recedere da contratti stipulati per pubblicare l'opera.\newline
    Il diritto di inedito si esaurisce con la pubblicazione dell'opera.\newline
    Se è fissato un termine per la pubblicazione le opere non possono essere pubblicate
    prima di tale scadenza
    \item Il diritto al \textbf{pentimento}: diritto di ritirare l'opera dal commercio in presenza
    di gravi ragioni morali (motivi etici, politici, religiosi).
    Può riguardare diverse versioni dell'opera e anche opere derivate dall'opera originale.\newline
    Per poterlo fare, tuttavia, l'autore ha l'obbligo di indennizzare coloro che hanno
    acquistato i diritti di utilizzazione economica (es. riprodurre, diffondere, eseguire, rappresentare
    l'opera medesima)
\end{itemize}

\subsection{Quali sono i diritti patrimoniali d'autore?}
Il titolare ha il diritto esclusivo di utilizzazione economica dell'opera nel suo insieme
e in ciascuna delle sue parti, ad esempio mediante:
\begin{itemize}
    \item Pubblicazione
    \item Riproduzione: diritto esclusivo di effettuare o autorizzare la moltiplicazione in copie, diretta o indiretta,
    temporanea o permanente, totale o parziale dell'opera, in qualunque modo o forma. \newline
    Sono esentate le riproduzioni temporanee prive di rilievo economico proprio, transitorie o accessorie,
    parte integrante ed essenziale di un procedimento tecnologico, effettuate all'unico scopo di consentire la trasmissione in rete tra terzi
    con l'intervento di un intermediario; è esentato l'utilizzo legittimo di un'opera.
    \item Distribuzione: diritto esclusivo di effettuare o autorizzare la messa in commercio o in circolazione
    o a disposizione del pubblico con qualsiasi mezzo e a qualsiasi titolo dell'originale dell'opera o di esemplari
    di essa (es. vendita, noleggio).\newline
    Questo diritto di esaurisce nella Comunità europea se la prima vendita o il primo atto di trasferimento è effettuato
    dal titolare o con il suo consenso. Significa che gli esemplari \textit{usati} dell'opera possono essere ulteriormente
    distribuiti da chi li ha acquistati in modo legale.
    \item Elaborazione
    \item Noleggio: il titolare ha il diritto esclusivo di noleggiare l'opera; azione intesa come il cedere in uso per
    un periodo di tempo per consentire un beneficio economico o commerciale diretto o indiretto
    \item Prestito: il titolare ha il diritto esclusivo di dare in prestito l'opera; azione intesa come cessione in uso fatta
    da istituzioni aperte al pubblico per un periodo di tempo limitato per usi diversi
    \item Comunicazione al pubblico: diritto esclusivo di effettuare o autorizzare la comunicazione mediante
    mezzi di diffusione a distanza. Comprende la messa a disposizione del pubblico in modo che l'utente possa avere
    accesso dal luogo e dal momento scelti individualmente (on demand).\newline
    Si dice che la comunicazione al pubblico \textbf{non esaurisce} il diritto per intendere che anche dopo
    che è avvenuta la comunicazione il diritto resta in capo al titolare.
    \item Diritto di traduzione e modificazione: diritto esclusivo di effettuare o autorizzare la traduzione in
    altra lingua, tutte le forme di modificazione, elaborazione e trasformazione dell'opera.
\end{itemize}

\subsection{I diritti patrimoniali devono essere trasferiti tutti insieme o possono essere ceduti separatamente?}
I diritti patrimoniali d’autore sono tra loro interdipendenti, pertanto l’autore può decidere di trasferire uno
o più diritti patrimoniali mantenendo però gli altri nella propria disponibilità. L’esercizio di uno dei diritti patrimoniali
da parte di un soggetto terzo non esclude quindi di per sé la possibilità di esercizio esclusivo di altri diritti
patrimoniali (non ceduti) da parte dell’autore. Proprio per questa ragione, il contratto di trasferimento va interpretato
restrittivamente: ciascun diritto diverso deve essere esplicitamente concesso dall'autore all’acquirente.

\subsection{Che cosa sono le libere utilizzazioni?}
Seguendo un principio generale, è necessario il consenso dell'autore allo sfruttamento e al godimento della sua opera. \newline
Vi sono però delle eccezioni alla regola generale, casi in cui l'utilizzazione è libera (casi tassativi).\newline
Le libere utilizzazioni non devono essere in contrasto con il normale sfruttamento dell'opera e non devono arrecare
ingiustificato pregiudizio al titolare dei diritti.\newline
Alcuni esempi di libere utilizzazioni sono:
\begin{itemize}
    \item Copia per uso personale di opere letterarie (con mezzi non idonei a diffusione dell'opera al pubblico;
    limite del 15\% di ciascun volume)
    \item Riassunto, citazione, riproduzione, comunicazione al pubblico (per scopi di critica, discussione, insegnamento).
    Obbligo di indicare la fonte
    \item Riproduzione di articoli di attualità, a meno che la riproduzione sia espressamente riservata
    \item Pubblicare musica e immagini degradate per uso didattico, scientifico, privo di scopi di lucro
\end{itemize}

\subsection{Che cosa sono i diritti connessi?}
I diritto connessi sono diritti patrimoniali su opere dell'ingegno riconosciuti dalla legge che spettano a \textbf{soggetti diversi
dall'autore}, ad esempio artisti, interpreti, emittenti radiofoniche e televisive ecc... \newline
Si tratta di soggetti che consentono al pubblico di fruire dell'opera. Questi soggetti sono titolari anche di diritti morali.

\subsection{Principio di esaurimento comunitario del software}
Il principio di esaurimento comunitario prevede che il diritto di distribuzione di un'opera originale o di una sua copia venga esaurito nella comunità europea se la prima vendita o il primo atto di trasferimento è effettuato dal titolare o con il suo consenso.
\newline
Dopo la prima vendita il titolare non potrà controllare l'ulteriore distribuzione di quell'esemplare, perde il diritto di privativa su quell'opera.
Per effetto dell'esaurimento il diritto del titolare non si esaurisce in maniera assoluta, ma soltanto su quel bene specifico, inteso come esemplare del suo prodotto, che sia stato immesso in commercio.
\newline
Nel caso di mancanza del principio di esaurimento il titolare di un marchio, ad esempio, potrebbe impedire che i propri beni approdino su un determinato mercato o che lo facciano solo ad un prezzo da lui scelto. Il principio funge, quindi, da catalizzatore della corretta concorrenza consentendo al titolare dei diritti di decidere riguardo alla distribuzione dei propri prodotti sul mercato ma impedendo che tale potere perduri nel resto della catena distributiva. Quest'ultima segue le regole riguardanti il principio della libera circolazione che si basano sulla valutazione del rapporto prezzo-quantità-qualità.


\subsection{Come è tutelato il software nell'ordinamento giuridico italiano?}
I programmi per elaboratore sono opere protette \textbf{in base alla disciplina sul diritto d'autore} (infatti, come vedremo parlando
di brevetti, non è possibile brevettare un software a sè stante).\newline
Il software è tutelato dal diritto d'autore in qualsiasi forma (codice sorgente o eseguibile), incluso il materiale
preparatorio. A tal fine deve essere \textbf{originale}, cioè risultato di creazione intellettuale
dell'autore (non mera copiatura).\newline
Non sono protette le idee e i principi alla base di qualsiasi elemento del software. \newline
\textbf{Viene invece tutelata la forma espressiva del software} (struttura e sviluppo delle istruzioni che compongono il programma).

\subsection{Chi è titolare dei diritti patrimoniali sul software? E dei diritti morali?}
Il titolare dei diritti \emph{patrimoniali} sul software varia a seconda dei casi e può trattarsi di:
\begin{itemize}
    \item Autore
    \item Altri soggetti (es. Pubbliche Amministrazioni)
    \item Il dipendente (es. di una software house)
\end{itemize}
Il diritto d'autore sul software è costituito da diritti morali, che rimangono sempre attribuiti all'autore
(inteso come persona fisica che scrive il programma), e dal diritto patrimoniale, che riguarda lo sfruttamento economico dell'opera
e che può essere oggetto di cessione.\newline
Una particolare attenzione è da porre al caso in cui un dipendente di un'azienda sviluppi un software. In questo caso si ha
che l'utilizzazione economica sul software è diritto esclusivo del \emph{datore di lavoro}; \textbf{questa situazione è valida a patto che
il software sia stato sviluppato nello svolgimento delle mansioni del dipendente e su istruzioni del datore di lavoro}. Modifiche a
queste condizioni possono tuttavia essere stabilite mediante contratto. \newline
Riguardo i diritti patrimoniali, se le operazioni di caricamento, visualizzazione, esecuzione o
memorizzazione richiedono la riproduzione del programma, anche tali operazioni sono soggette all'autorizzazione del titolare del diritto d'autore.
Inoltre, viene riservata all'autore del software qualsiasi forma di distribuzione al pubblico, la traduzione, l'adattamento, la trasformazione e ogni
altra modificazione del programma.

\subsection{Qual è la differenza tra opera collettiva e opera in comunione?}
Accade spesso che più autori decidano di collaborare al fine di realizzare un'opera. Quest' ultima, dal punto
di vista legale, può essere riconosciuta come:
\begin{itemize}
    \item Opera collettiva: opera facilmente scindibile; i diritti morali appartengono ai singoli autori, ciascuno per la propria parte -
    mentre i diritti morali dell'opera nel complesso sono di chi l'ha diretta / organizzata / curata. \newline
    I diritti patrimoniali appartengono all'editore (salvo diverso patto).
    \item Opera in comunione: opera che non può essere scissa (il contributo dei singoli autori è indistinguibile).\newline
    Il diritto spetta in comune a tutti gli autori per l'assunto che tutte le parti indivise sono di uguale valore.\newline
    È necessario l'accordo di tutti i coautori per la pubblicazione dell'opera inedita, modifica, utilizzo in forma diversa
    da quella della prima pubblicazione. In caso di un ingiustificato rifiuto tali atti possono essere autorizzati dall'autorità giudiziaria.\newline
    I diritti morali si esercitano, però, individualmente. \newline
    Come afferma l'articolo 1101 del Codice Civile, il concorso dei partecipanti, tanto nei vantaggi quanto nei pesi della comunione,
    è in proporzione delle rispettive quote.
\end{itemize}

\subsection{In cosa consistono i diritti esclusivi sul software?}
I diritti esclusivi sul software, spettanti a colui che lo ha registrato, consistono principalmente nel diritto di effettuare ed
autorizzare: la riproduzione, permanente o temporanea, totale o parziale, del programma con qualsiasi mezzo o in qualsiasi forma;
la traduzione, l'adattamento, la trasformazione e ogni altra modificazione del software, nonchè la riproduzione dell'opera che ne
risulti, senza pregiudizio dei diritti di chi modifica il programma; qualsiasi forma di distribuzione al pubblico, compresa la locazione, del
programma originale o di copie dello stesso.

\subsection{Quali sono le eccezioni per interoperabilità?}
Non è necessaria autorizzazione del titolare se riproduzione del codice e traduzione della sua forma effettuate per modificare la forma del codice
sono indispensabili per ottenere informazioni necessarie per l'interoperabilità con altri software di un programma creato
autonomamente. Questo è possibile a patto che:
\begin{itemize}
    \item Le attività sono compiute dal licenziatario o soggetto autorizzato
    \item Le informazioni non sono facilmente e rapidamente accessibili
    \item Le attività sono limitate alle parti di software necessarie per ottenere interoperabilità.
\end{itemize}
Le informazioni estratte per ottenere interoperabilità non possono essere utilizzate a fini diversi dal conseguimento di interoperabilità. Qualunque clausola contrattuale
contraria a questa disposizione per interoperabilità è \emph{nulla}.

\subsection{Dopo aver sviluppato un software è necessario registrarlo? La registrazione è utile?}
La registrazione di un software sul Pubblico Registro Software presso la SIAE è:
\begin{itemize}
    \item facoltativa: \emph{non è necessaria} per godere dei diritti d'autore, che sorgono con la creazione dell'opera,
    ma \emph{può essere utile per fornire una prova dell'esistenza del software e della titolarità dei diritti}
    \item a pagamento: costa 126,62 euro
\end{itemize}
Fa fede fino a prova contraria dell'esistenza del software, del suo autore e del titolare dei diritti (se il soggetto è diverso),
fornisce una prova documentale dell'esistenza del software alla data di deposito; quest' ultimo ha validità \emph{quinquennale}. Può
essere rinnovato per la stessa durata.\newline
Nel processo di registrazione la SIAE inserisce nel registro i dati dichiarati apponendo un numero progressivo e la data agli esemplari (conservati nei suoi archivi)
del software. Fornisce al richiedente un attestato di registrazione.\newline
Il deposito del software alle SIAE non dà diritto all'iscrizione alla SIAE o per la tutela del software
da parte della SIAE.

\subsection{Brevettabilità del software}
Citando l'Art. 52 sulle Invenzioni brevettabili della Convenzione di Monaco sul brevetto europeo:\newline
"I brevetti europei sono concessi per le invenzioni in ogni campo tecnologico, a condizione che siano nuove,
implichino un'attività inventiva e siano atte ad avere un'applicazione industriale.\newline
\textbf{Non sono considerate invenzioni ai sensi del paragrafo precedente in particolare i programmi informatici}
(come le scoperte, le teorie scientifiche, le creazioni estetiche ecc...). \newline
\textbf{La brevettabilità è esclusa solo nella misura in cui il brevetto concerne scoperte, teorie, piani, princìpi,
metodi, programmi e presentazioni di informazioni considerati in quanto tali}.\newline
\newline
\textbf{Il software è tutelato dal diritto d'autore}. Tuttavia l'Ufficio europeo dei brevetti e gli uffici nazionali
hanno concesso numerosi \textbf{brevetti per invenzioni che utilizzano software}.

\subsection{Misure tecnologiche a protezione del software}
Gli autori possono tutelare i propri diritti anche utilizzando misure tecnologiche. \newline
I titolari di diritto d'autore e connessi possono apporre sulle opere o sui materiali
protetti misure tecnologiche di protezione efficaci. Esempi sono la cifratura, distorsione,
meccanismo di controllo delle copie al fine di impedire o limitare atti non autorizzati dai titolari
dei diritti (es. accesso, copia).

\subsection{Le principali sanzioni a tutela delle opere dell'ingegno}
Le principali sanzioni a tutela delle opere dell'ingegno si suddividono in \emph{civili, penali, amministrative}:
\begin{itemize}
    \item Sanzioni civili: chi teme violazione di un diritto di utilizzazione economica o vuole impedire
    continuazione o ripetizione di una violazione può agire in giudizio per accertare il diritto e vietare
    il proseguimento della violazione. \newline
    Chi è leso nell'esercizio di un diritto di utilizzazione economica può agire in giudizio per ottenere il risarcimento
    del danno e la distruzione o rimozione dello stato di fatto da cui risulta la violazione.
    \item Sanzioni penali (art. 171): multa da euro 51 a euro 2065 per chiunque, senza averne diritto, a qualsiasi
    scopo e in qualsiasi forma riproduce, diffonde, vende, mette a disposizione del pubblico, un' opera dell'ingegno
    altrui o riproduce un numero di esemplari maggiore di quello che aveva il diritto di riprodurre.\newline
    È sanzionato penalmente chiunque abusivamente duplica per trarne profitto programmi per elaboratore o ai medesimi fini
    importa, distribuisce, vende, detiene a scopo commerciale o imprenditoriale o concede in locazione programmi contenuti
    in supporti non contrassegnati dalla SIAE.
    L'articolo 171-terzo sanziona penalmente numerosi comportamenti, a condizione che il fatto sia commesso per
    uso non personale e a fini di lucro.  \emph{Pena: reclusione da 6 mesi a 4 anni e multa da 2582 a 15492 euro. Pene
    maggiori se fatto è di rilevante gravità}.\newline
    Alcuni esempi di comportamenti sanzionati:
    \begin{itemize}
        \item Ritrasmissione o diffusione in assenza di accordo con il legittimo distributore, con qualsiasi mezzo
        un servizio criptato ricevuto per mezzo di apparati atti alla decodificazione di trasmissioni ad accesso condizionato.
        \item Rimozione o alterazione abusiva di informazioni elettroniche o distribuzione di materiali protetti dai quali sono
        state rimosse o alterate le informazioni elettroniche stesse.
    \end{itemize}
    \item Sanzioni amministrative (art. 174 [bis/ter]): Sanzione amministrativa pecuniaria pari al doppio
    del prezzo di mercato dell'opera o del supporto oggetto della violazione, in misura comunque non inferiore a 103,00 euro.\newline
    La sanzione amministrativa si applica nella misura stabilita per ogni violazione e per ogni esemplare abusivamente duplicato
    o riprodotto.\newline
    Le sanzioni amministrative pecuniarie possono essere accompagnate da sanzioni accessorie (confisca del materiale,
    pubblicazione del provvedimento su un quotidiano a diffusione nazionale)
\end{itemize}
