\newpage
\section{Tutela giuridica delle banche di dati}

\subsection{Cosa è una banca di dati?}
"Una banca di dati è una raccolta di opere, dati o altri elementi indipendenti sistematicamente
o metodicamente disposti ed individualmente accessibili mediante mezzi elettronici o in altro
modo"

\subsection{Quali banche di dati possono essere protette in base alla normativa sul diritto d'autore?}
Sono protette dal diritto d'autore le "banche di dati che per la scelta o la disposizione
del materiale costituiscono creazione intellettuale dell'autore"

\subsection{Che cosa è tutelato della banca di dati?}
Della banca di dati è tutelata la forma espressiva, cioè la struttura, la disposizione dei contenuti,
la \emph{scelta} dei contenuti.\newline
La tutela non si estende al contenuto della banca di dati. Infatti restano impregiudicati i diritti sul contenuto
della banca di dati.\newline
Si precisa, inoltre, che la tutela non si estende al software utilizzato per la costruzione e il
funzionamento delle banche di dati elettroniche.

\subsection{Il concetto di creatività/originalità}
Una banca di dati può includere materiale originale (originalità intrinseca del contenuto).\newline
Una banca di dati può includere materiale non originale (è protetta se la disposizione dei dati
è originale).\newline
\textbf{L'ordine cronologico o alfabetico non è considerato originale poichè manca il requisito
della creatività}

\subsection{Banca di dati selettiva e non selettiva}
Una banca di dati \emph{selettiva} è una banca di dati il cui contenuto è scelto discrezionalmente
dall'autore; nell'esercizio di questa discrezionalità di scelta \emph{è ravvisabile l'originalità}.\newline
Una banca di dati \emph{non selettiva} è una banca di dati che include tutti i dati possibili su un determinato
argomento (i dati non sono selezionati); la creatività va ricercata nella struttura della banca di dati.

\subsection{Quando sorge il diritto d'autore sulla banca di dati?}
Il diritto si acquista con la creazione dell'opera (da tener conto che
spesso la banca di dati è un' opera collettiva o in comunione)

\subsection{Quali sono i diritti esclusivi dell'autore di una banca di dati?}
L'autore ha il diritto esclusivo di:
\begin{itemize}
    \item riproduzione permanente o temporanea, totale o parziale, con qualsiasi mezzo in qualsiasi forma;
    \item traduzione, adattamento, disposizione diversa, qualsiasi modifica
    \item qualsiasi forma di distribuzione al pubblico dell'originale o di copie della banca di dati
    \item qualsiasi presentazione, dimostrazione, comunicazione in pubblico della banca di dati, trasmissione con qualsiasi
    mezzo e in qualsiasi forma
\end{itemize}

\subsection{Quali sono le deroghe al diritto d'autore su di una banca di dati?}
Alcune attività non sono soggette ad autorizzazione del titolare. Ad esempio l'accesso o la consultazione
per scopi didattici e di ricerca scientifica, non svolta nell'ambito di un' impresa (scopo non commerciale).
In questi casi è necessario indicare la fonte oltre al fatto che deve trattarsi di una riproduzione non permanente.\newline
Ulteriore caso si presenta nell'impiego di una base di dati per scopi di sicurezza pubblica o per effetto di una
procedura amministrativa o giurisdizionale.

\subsection{Cosa può fare un utente legittimo che utilizza una banca di dati?}
Se è necessario l'accesso al contenuto della base di dati al fine del suo normale impiego, l'utente legittimo
può, senza autorizzazione del titolare, effettuare le seguenti operazioni:
\begin{itemize}
    \item riproduzione permanente o temporanea, totale o parziale, con qualsiasi mezzo e in qualsiasi forma
    \item presentazione, dimostrazione o comunicazione in pubblico della banca di dati, trasmissione con qualsiasi mezzo
    e in qualsiasi forma
    \item riproduzione, distribuzione, comunicazione, presentazione o dimostrazione in pubblico dei risultati
    delle operazioni di traduzione, adattamento, modifica
\end{itemize}
Se l'utente è autorizzato a usare solo una parte della banca di dati, queste attività potranno essere legittimamente effettuate
solo su tale parte. \newline
Clausole contrattuali che vietino queste attività sono \emph{nulle}.

\subsection{In cosa consiste il diritto di costitutore?}
Il diritto del costitutore consiste nel diritto di \textbf{vietare operazioni di estrazione e reimpiego dell'intera banca di dati
o di una parte sostanziale di essa}.\newline
Sono vietate anche attività di estrazione e reimpiego di parti non sostanziali della banca di dati
"\emph{ripetute e sistematiche}" se
tali operazioni sono contrarie alla normale gestione della banca di dati o se arrecano ingiustificato pregiudizio al
costitutore.\newline
Il diritto del costitutore, detto anche \textbf{diritto sui generis}, spetta al soggetto che ha effettuato un
investimento rilevante per realizzare, verificare o presentare una banca di dati.\newline
Per investimento intendiamo un impegno finanziario, di tempo, di lavoro; esso deve essere rilevante
qualitativamente e quantitativamente. Lo si valuta in relazione al settore, agli investimenti normalmente effettuati
da altri o dallo stesso costitutore.\newline
Il diritto del costitutore è un diritto \textbf{indipendente e parallelo al diritto d'autore} sulla banca di dati. Ha la natura di un \textbf{diritto connesso}.\newline
Il costitutore deve essere un cittadino o residente dell'Unione Europea o, se società o impresa, essere
stabilito nell'Unione Europea.\newline
Una particolare attenzione va posta al fatto che questo diritto non include requisiti di creatività o originalità.
Il diritto di costitutore può essere anche associato ad un investimento su una banca di dati non originale e quindi non
tutelabile dal diritto d'autore.\newline

\subsection{Quando dura il diritto del costitutore?}
Il diritto del costitutore sorge al momento del completamento della banca di dati e ha durata di 15 anni
(a partire dal 1º gennaio dell'anno successivo al completamento della banca di dati).\newline
Si pone particolare attenzione al fatto che per le banche di dati messe a disposizione del pubblico prima della
scadenza di questo periodo, il diritto si esaurirà il 15º anno dal 1º gennaio dell'anno
successivo alla prima \textbf{messa a disposizione del pubblico}

\subsection{Vi sono eccezioni e limitazioni al diritto del costitutore?}
Le eccezioni e limitazioni alle privative previste dalla disciplina del diritto d'autore (libere utilizzazioni) si
applicano anche al diritto del costitutore. Infatti il Decreto Legislativo 68/2003 sancisce espressamente
l'applicabilità delle eccezioni e delle limitazioni anche al diritto del costitutore.

\subsection{Principio dell'esaurimento comunitario delle banche di dati}
La prima vendita dell'originale di una banca di dati o di una sua copia nell'Unione Europea da parte del titolare
del diritto o di un soggetto da questi autorizzato esaurisce il diritto di controllare la sua vendita successiva
nell'Unione Europea. Questo diritto riguarda sia il diritto d'autore sia il diritto del costitutore. \newline
\textbf{Diversa situazione si ha nel caso la banca di dati sia trasmessa on line: si tratta di una presentazione di servizi e
non di consegna di beni}.

\subsection{Cosa succede se vengono effettuati nuovi investimenti sulla banca di dati?}
Se sono apportate modifiche o integrazioni sostanziali sulla banca di dati che richiedono nuovi investimenti
rilevanti, si inaugura un nuovo termine di tutela del diritto del costitutore. Quindi si pone una nuova durata di
15 anni dal completamento della banca di dati modificata o dalla sua messa a disposizione del pubblico.

\subsection{Creazione di un sito web e aspetti legali da considerare}
Il sito web può contenere opere dell'ingegno creative protette dal diritto d'autore come testi, fotografie,
disegni, video, file audio, ecc. \newline
Non è altrettanto semplice affermare se il sito web come entità nel suo complesso sia o meno tutelabile dal
diritto d'autore. A tal proposito dottrina e giurisprudenza affermano che se il sito web è creativo e originale
può essere considerato opera dell'ingegno. Infatti una sentenza del tribunale di Bari del 1998 afferma che: \newline
\emph{un'opera telematica è meritevole di tutela se le modalità d'accesso, il tipo di informazioni e i modi di consultazione
sono originali e frutto di un'attività intellettuale di tipo creativo}.\newline
La riproduzione di informazioni e notizie è lecita purchè non sia effettuata con l'impiego di atti contrari
agli usi onesti in materia giornalistica e purchè se ne citi la fonte. \newline
È illecito riprodurre o radiodiffondere senza autorizzazione i bollettini di informazioni delle agenzie giornalistiche o di informazioni, prima che
siano trascorse 16 ore dalla diramazione del bollettino e, comunque prima della loro pubblicazione in un giornale
o altro periodico autorizzato dall'agenzia. I bollettini devono contenere l'esatta indicazione del giorno e dell'ora della
diramazione. \newline
È inoltre illecita la riproduzione sistematica di informazioni o notizie, pubblicate o radiodiffuse, a fine di lucro,
sia da parte di giornali o altri periodici, sia da parte di imprese di radiodiffusione. C'è da porre attenzione al fatto
che finalità di lucro possono sussistere anche se non sono forniti servizi a pagamento.\newline
\newline
Possono costituire atti di concorrenza sleale:
\begin{itemize}
    \item l'imitazione servile del sito di un concorrente
    \item l'uso nelle pagine web di "meta tags" corrispondenti al nome di un'impresa concorrente,
    per far comparire, tra i risultati di ricerca, il proprio sito e dunque approfittare della notorietà
    raggiunta nel settore dall'impresa concorrente.
\end{itemize}

\subsection{Ulteriori aspetti legali nella creazione di un sito web}
Per capire come costruire un sito web a norma di legge occorre tenere conto di molti altri aspetti, come ad esempio:
\begin{itemize}
    \item le regole sul trattamento di dati personali (per esempio se si raccolgono dati degli utenti
    mediante un modulo online, se si inviano newsletter, i cookies, ecc.)
    \item le regole sul commercio elettronico tra imprese (es. obblighi di informazione del venditore) e le regole
    particolari a tutela del consumatore (es. diritto di recesso)
    \item le informazioni obbligatorie che le imprese devono inserire nei siti web (es. Partita IVA, informazioni aziendali,
    ecc.)
\end{itemize}
