\subsection{Cosa si intende per documento? Quali sono le varie tipologie di documento?}
Per \textit{documento} l’ordinamento giuridico italiano non prevede una definizione in generale;
il Codice civile definisce e regola alcuni tipi specifici di documenti:
\begin{itemize}
    \item scritture private;
    \item riproduzioni meccaniche;
    \item atti pubblici.
\end{itemize}
\subsection{Cosa si intende per scrittura privata?}
La \textit{scrittura privata} fa piena prova della provenienza delle dichiarazioni da chi l’ha
sottoscritta se colui contro cui la scrittura è prodotta ne riconosce la sottoscrizione,
ovvero se questa è legalmente considerata come riconosciuta.
\newline Occorre fare una querela di falso del documento per smentire tale prova.
\newline La firma è considerata valida se è autenticata e non viene disconosciuta durante una causa.
\subsection{Cosa si intende per riproduzione meccanica?}
Per Riproduzione Meccanica si intende ogni rappresentazione meccanica di fatti e di cose
(riproduzioni fotografiche, informatiche o cinematografiche, registrazioni fonografiche, etc.),
le quali formano piena prova dei fatti e delle cose rappresentate, a meno che non vengano disconosciute
nella loro conformità da colui contro il quale sono state prodotte.
\subsection{Cosa si intende per atto pubblico?}
Un Atto Pubblico è un documento redatto da un notaio o altro pubblico ufficiale autorizzato
ad attribuirgli fede nel luogo dove l’atto è formato.
Questo documento dà certezza ufficiale su tutto ciò che si è svolto davanti al pubblico ufficiale,
contestabile solo con querela di falso (es: atto di compravendita immobiliare, stipula contratto di mutuo, etc.).
\subsection{Cosa si intende per firma elettronica?}
La FIRMA ELETTRONICA consiste in dati in forma elettronica, acclusi o connessi tramite associazione
logica ad altri dati elettronici e utilizzati dal firmatario (persona fisica) per firmare.
A una firma elettronica non può essere negato alcun effetto giuridico e l’ammissibilità
come prova in procedimenti giudiziari per il solo motivo della sua forma elettronica
o perché non soddisfa i requisiti per firme elettroniche qualificate.
\newline
La FIRMA ELETTRONICA AVANZATA è una firma elettronica più sicura dato che garantisce l’autenticità:
connessa unicamente al firmatario, idonea a identificare il firmatario,
creata mediante dati per l’apposita creazione che il firmatario può usare sotto
il proprio esclusivo controllo e collegata a dati sottoscritti in modo da
consentire l’identificazione di ogni futura modifica.
\newline
La FIRMA ELETTRONICA QUALIFICATA è una firma elettronica avanzata che viene
creata da un dispositivo apposito ed è basata su un certificato qualificato
(se rilasciato in uno Stato membro, questo certificato permette di riconoscere
come firma elettronica qualificata valida in tutti gli Stati membri).
Questa firma ha effetti giuridici equivalenti a quelli di una firma autografa.
\newline
La FIRMA DIGITALE è una particolare firma elettronica qualificata basata
su un sistema di chiavi crittografiche (una pubblica e una privata) per garantire
integrità e provenienza di un documento: questa deve riferirsi in modo univoco
a un solo soggetto e al documento a cui è apposta o associata.
Per generarla deve essere usato un certificato valido non scaduto, non revocato
e non sospeso al momento della firma; dal certificato qualificato devono risultare validità,
estremi identificativi del titolare e del certificatore ed eventuali limiti d’uso.
Firmare con una firma basata su un certificato qualificato scaduto,
revocato, non valido o sospeso equivale a una mancata sottoscrizione;
revoca o sospensione hanno effetto dal momento di pubblicazione, se motivate.
Le regole della firma digitale si applicano anche per firme basate su certificati
di Stati non membri se il certificato è garantito da un certificatore stabilito
nell’Unione Europea, se il certificatore possiede i requisiti previsti da eIDAS
ed è qualificato in uno Stato membro oppure se il certificato o il certificatore
è riconosciuto in forza di un accordo bilaterale o multilaterale con l’Unione Europea.
Le firme elettroniche qualificate e digitali, anche se scaduto, revocato o sospeso il
relativo certificato, sono valide se sono state generate in un momento precedente alla
revoca, scadenza o sospensione del certificato.
\subsection{Che cos’è un certificato qualificato? Quali sono le principali informazioni che contiene?}
Un certificato qualificato è un certificato di firma elettronica (ovvero un attestato
elettronico che collega dati di convalida di una firma elettronica a una persona fisica
e conferma almeno il nome o lo pseudonimo di tale persona) che è stato rilasciato da un
prestatore di servizi fiduciari qualificato e conforme ai requisiti.
Le principali informazioni che contiene sono il nome o pseudonimo del firmatario,
l’inizio e la fine (estremi) del periodo di validità del certificato e la firma elettronica
avanzata o il sigillo elettronico avanzato del prestatore di servizi fiduciari qualificato che rilascia il certificato.
\subsection{Il funzionamento della firma digitale}
La firma digitale funziona come una firma elettronica qualificata,
ma basandosi su un sistema a doppia chiave crittografica, e integra e sostituisce
sigilli, punzoni, timbri, contrassegni e marchi.
Questa è risultato di una procedura informatica che si basa sulla tecnica
di crittografia asimmetrica o a doppia chiave pubblica-privata: queste due chiavi
sono diverse e necessitano di funzionare congiuntamente.
\newline Il mittente firma il documento informatico con la sua chiave privata,
di cui solo lui è in possesso, e il destinatario verifica il documento con la chiave pubblica
del mittente, garantendo la paternità e l’integrità del documento.
Il meccanismo è applicato esternamente al documento, come una sua impronta digitale,
fingerprint, una stringa di dati che sintetizza in modo univoco il contenuto:
questa viene generata con algoritmi di hash, i quali rendono quasi impossibile
ottenere una stessa impronta da due file diversi (integrità del file). Tutti i
legali rappresentati della società sono dotati di firma digitale (es:
avvocati), ma di base chiunque può ottenerne una.

\subsection{La firma grafometrica è una firma elettronica avanzata? Perché?}
La firma grafometrica risponde pienamente ai requisiti di firma elettronica;
inoltre è catalogabile come firma elettronica avanzata poiché viene raccolta
seguendo un procedimento che parte dall’identificazione del firmatario fino a
una garanzia d’integrità del documento a garanzia dello stesso.
\newline
È a ogni effetto una firma biometrica poiché si ricollega a precisi e personali tratti biometrici
(comportamentali) della persona.

\subsection{Che cosa succede se firmo un documento e il certificato è scaduto?}
Se firmo un documento e, al momento della firma, il certificato è scaduto, equivale a una mancata sottoscrizione.

\subsection{La firma autenticata}
La Firma Autenticata è una firma elettronica o firma elettronica avanzata che richiede
l’intervento di un pubblico ufficiale (in particolare notaio) che la autentichi:
questi si accertano dell’identità del firmatario, attestano che la firma è stata apposta in sua presenza,
accertano la validità dell’eventuale certificato elettronico e, infine, accertano che il documento sottoscritto
non sia contrario all’ordinamento giuridico.

\subsection{Definizione di documento informatico. Un documento informatico costituisce “forma scritta”? A quali condizioni?}
Il documento informatico è un documento elettronico, ovvero qualsiasi
contenuto conservato in forma elettronica, che contiene la rappresentazione informatica
di atti o dati giuridicamente rilevanti. Molti pensano che un documento elettronico
non possa avere validità giuridica, soprattutto se sprovvisto di firma elettronica,
ma ciò è sbagliato perché non si può negare al documento elettronico effetti giuridici
e l’ammissibilità come prova in procedimenti giudiziari solo per via della sua forma elettronica.
Il documento informatico soddisfa il requisito della forma scritta e ha l’efficacia di scrittura privata,
a patto che vi sia apposta una firma digitale (o firma elettronica avanzata o firma elettronica qualificata)
oppure che sia formato, previa identificazione informatica dell’autore, attraverso
un processo avente i requisiti stabiliti dalle Linee Guida al fine di garantire sicurezza,
integrità e immodificabilità del documento e la riconducibilità inequivocabile ed esplicita all’autore.
\newline In ogni altro caso, l’idoneità alla forma scritta e il valore probatorio di un documento informatico
sono liberamente valutabili in giudizio in relazione alla sua sicurezza, integrità e immodificabilità.
Inoltre, se apposte in conformità alle Line Guida, data e ora di formazione del documento sono opponibili a terzi.
Un documento informatico, sottoscritto anche con firma elettronica qualificata o digitale, non soddisfa
in alcun modo il requisito di immodificabilità se contiene macroistruzioni, codici eseguibili o altri
elementi che attivino funzioni in grado di modificare gli atti, i fatti o i dati rappresentati nello stesso documento.

\subsection{Come faccio ad assicurare la continuità nel tempo degli effetti giuridici del documento informatico?}
Per assicurare la continuità nel tempo degli effetti giuridici del documento informatico
devo necessariamente rispettare gli obblighi ti conservazione e di esibizione dei documenti:
questi sono soddisfatti se le procedure relative sono effettuate in modo tale da garantire
conformità ai documenti originali e alle Linee Guida.
\newline
Se il documento è conservato per legge da uno dei soggetti indicati all’articolo 2
comma 2 (es: pubbliche amministrazioni e gestori di servizi pubblici), allora cessa l’obbligo
da parte dei cittadini e delle imprese di conservazione a loro carico; le amministrazioni
rendono disponibili a tali cittadini e imprese i documenti conservati tramite servizi online.

\subsection{Trasmissione del documento informatico. Quand'è che un documento informatico si considera inviato/pervenuto al destinatario?}
Un documento informatico inviato telematicamente si intende spedito dal mittente
se inviato al proprio gestore e consegnato (pervenuto) al destinatario se reso disponibile
all’indirizzo di posta elettronica del destinatario, dunque non se aperto e/o letto dal destinatario,
nella casella di posta messa a disposizione dal gestore. I documenti trasmessi da chiunque
ad una Pubblica Amministrazione, con qualsiasi mezzo telematico o informatico idoneo ad accertarne la provenienza,
soddisfano il requisito della forma scritta e la loro trasmissione non deve essere seguita
da quella del documento originale. L’invio telematico di comunicazioni che necessitano di
ricevuta di invio e consegna avviene mediante PEC (Posta Elettronica Certificata)
o altre soluzioni tecnologiche simili individuate con le regole tecniche:
la trasmissione del documento informatico mediante PEC equivale,
salvo casi particolari indicati dalla legge, alla notificazione a mezzo posta;
data e ora d’invio e recezione di documento informatico mediante PEC sono opponibili a terzi se conformi alla normativa in vigore.

\subsection{Che cos'è la posta elettronica certificata? Qual è la sua validità giuridica?}
La posta elettronica certificata, nota come PEC, è una soluzione tecnologica per
la trasmissione di documenti elettronici.
È particolarmente interessante e utile sotto vari aspetti, dal momento che
permette un sistema informatico equivalente alla posta con ricevuta di invio e di consegna,
salvo casi specificati dalla legge. A livello giuridico, data e ora d’invio e recezione
di documento informatico mediante PEC sono opponibili a terzi se conformi alla normativa in vigore.

\subsection{Quali sono i principali compiti dei certificatori?}
I certificatori sono le “terze parti fidate” che intervengono
per garantire la reale identità del firmatario. I loro principali compiti consistono nel
verificare l’identità del soggetto, associarla a una chiave pubblica di cifratura,
attestare tali informazioni mediante l’emissione di un certificato e pubblicare tempestivamente revoca
e sospensione del certificato in apposite liste.


\subsection{Responasbilità dei certificatori}
Le responsabilità dei certificatori che cagionano danno ad altri svolgendo la propria
attività sono tenuti al risarcimento, se non provano di aver adottato tutte
le misure idonee ad evitare tal danno. Il certificato qualificato può contenere limiti
d’uso o valore purché questi siano riconoscibili da parte dei terzi e chiaramente
evidenziati nel certificato: qualora i terzi utilizzassero un certificato qualificato
eccedendo i limiti imposti e causando danno, il certificatore è esonerato dalla responsabilità di tali danni.


\subsection{Obblighi del titolare del certificato}
Il titolare del certificato è obbligato ad: assicurare
la custodia del dispositivo di firma o degli strumenti di autenticazione
informatica per l’utilizzo del dispositivo di firma da remoto;
adottare tutte le misure organizzative e tecniche idonee ad evitare danno ad altri; infine,
utilizzare personalmente il dispositivo di firma.

\subsection{Che cosa succede se un documento informatico contiene macroistruzioni o codice eseguibile?}
Qualora un documento informatico contenesse macroistruzioni o codice eseguibile, esso non soddisferebbe
il requisito di immodificabilità del documento e, dunque, perderebbe di valenza giuridica.

\subsection{Come devono essere conservate le chiavi? È lecito fare il backup
della chiave privata? Se temo di perdere la mia chiave posso chiedere al
certificatore di tenerla temporaneamente in deposito?}
Le chiavi private possono essere conservate in un dispositivo di firma, ma la chiave privata
e il dispositivo non possono essere duplicati (dunque non è lecito avere copie di backup)
e devono essere conservati con diligenza al fine di garantire integrità e riservatezza.
\newline Le informazioni di abilitazione alla chiave devono essere conservate in un luogo diverso dal dispositivo.
È necessario infine richiedere la revoca immediata se si è perso il possesso del dispositivo contenente la chiave privata
o se si teme sia stato utilizzato da non autorizzati. In base all'art.7 del DPR 513/97,
si può ottenere il deposito in forma segreta della chiave privata presso un notaio
o altro pubblico depositario autorizzato, che non può agire, però, anche come certificatore
(quest’ultimo infatti deve gestire solo la chiave pubblica di un utente).


\subsection{Per quanto tempo resta valido un certificato?}
Un certificato resta valido per un lasso di tempo predeterminato dal certificatore
al momento del rilascio, in funzione della robustezza crittografica delle chiavi:
è l’AGenzia per l’Italia Digitale (AGID) a determinare il periodo massimo di validità
del certificato qualificato. Ovviamente si parla in un periodo massimo di validità
poiché questo può essere revocato o sospeso anticipatamente rispetto alla scadenza.

\subsection{Revoca e sospensione del certificato}
Il certificato qualificato deve essere, a cura del certificatore: revocato o sospeso
per intervento delle autorità; revocato in caso di cessazione dell’attività del certificatore
(a meno che questi non designi un certificatore sostitutivo); revocato o sospeso a seguito
di richiesta del titolare o di un terzo designato dal titolare; revocato o sospeso in presenza
di cause limitative alla capacità del titolare, abusi o falsificazioni; revocato o sospeso
in casi previsti dalle Linee Guida che violano le regole tecniche.
\newline
La revoca o sospensione del certificato qualificato, indipendentemente dalla causa,
ha effetto dal momento della pubblicazione della lista che lo contiene, il quale deve essere attestato mediante riferimento temporale.

\subsection{Accesso ai certificati. Tutte le liste di certificati (validi/revocati/sospesi) devono necessariamente essere pubblicate?}
Le liste dei certificati revocati e sospesi devono essere rese pubbliche, mentre i certificati qualificati validi
possono essere resi accessibili al pubblico solo su richiesta del titolare al fine di verificarne le firme digitali.
Queste liste possono essere utilizzate solo a scopo di applicazione delle norme
sulla verifica e validità delle firme elettroniche qualificate e digitali.
Chiunque ha diritto di sapere se sia stato rilasciato un certificato qualificato a suo nome.
