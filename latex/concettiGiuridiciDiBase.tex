\newpage
\section{Concetti giuridici di base}

\subsection{Che cos'è una norma giuridica? E una sentenza?}
Una norma giuridica è l'unità elementare del sistema del diritto.
Essa è un comando o precetto generale ed \textbf{astratto}, che impone o proibisce
un certo comportamento, stabilisce diritti, obblighi, oneri, ecc. Si dice che
svolge una funzione precettiva. \newline
Leggi, decreti, direttive, regolamenti, ecc. sono divisi in articoli, numerati
progressivamente; gli articoli possono essere divisi in commi. \newline
Un comma può contenere una o più norme. \newline
Una sentenza è una decisione del giudice chiamato a decidere su una controversia;
è presa interpretando e applicando le norme. \newline
Una sentenza è un comando:
\begin{itemize}
    \item \textbf{individuale}: è rivolto alle parti in causa e non a tutti
    \item \textbf{concreto}: è relativo ad una determinata controversia
\end{itemize}

\subsection{I precedenti giurisprudenziali sono vincolanti in Italia? E nei paesi di common law?}
I precedenti giurisprudenziali \textbf{non} sono \textbf{vincolanti} in Italia (paese di civil law).
L'interpretazione di una norma data dal giudice vale solo per il caso concreto
con quella sentenza; tuttavia se più giudici danno un'interpretazione uniforme
di una norma è plausibile che in futuro altri possano dare la stessa interpretazione.
\newline
\newline
I precedenti giurisprudenziali sono invece vincolanti nei paesi di \textbf{common law}.
Nel diritto anglosassone, infatti, vale il principio del \textbf{precedente giudiziario vincolante}.

\subsection{Fonti del diritto}

Le fonti del diritto seguono una delineazione gerarchica tale che le fonti di
grado superiore vincolano le fonti di grado inferiore: le fonti di grado inferiore
che contrastano con quelle di grado superiore sono illegittime.\newline
Sistema gerarchico delle fonti del diritto:
\begin{itemize}
    \item Trattati della Comunità europea: accordi vincolanti tra paesi membri dell'UE. Definiscono
        le regole di funzionamento delle istituzioni europee, le procedure per l'adozione
        delle decisioni e le relazioni tra l'UE e i suoi paesi membri.\newline
        Essi si dividono in:
        \begin{itemize}
            \item Regolamenti comunitari: discipline dettagliate, direttamente applicabili negli Stati 
                membri, appena entrano in vigore devono essere rispettate all'interno dei singoli Stati.
            \item Direttive comunitarie: dettano obiettivi, principi e regole generali. Gli Stati
                membri devono dare attuazione alle direttive disciplinando la materia nei dettagli 
                con la propria legge interna, entro il termine stabilito nella direttiva stessa.
        \end{itemize}
    \item Costituzione e leggi costituzionali.\newline
        La Costituzione è la legge fondamentale della Repubblica. Essa è un'entità rigida: occorre
        un procedimento speciale (c.d. revisione costituzionale) per modificarla.\newline
        La Costituzione è composta da 139 articoli e relativi commi, suddivisi in varie sezioni:
        \begin{itemize}
            \item Principi fondamentali
            \item Parte I: Diritti e doveri dei cittadini
                \begin{itemize}
                    \item Titolo I: Rapporti civili
                    \item Titolo II: Rapporti etico-sociali
                    \item Titolo III: Rapporti economici
                    \item Titolo IV: Rapporti politici
                \end{itemize}
            \item Parte II: Ordinamento della Repubblica
        \end{itemize}
        Le leggi costituzionali sono leggi emanate in materie per cui
        la Costituzione prevede una "riserva di legge costituzionale". Alcune materie possono essere regolate
        solo con legge costituzionale. \newline
        Un esempio è l'Art. 116 della Costituzione: Regioni autonome: i loro
        statuti speciali sono adottati con legge costituzionale. Oppure l'Art. 137 che afferma
        che i giudizi di legittimità costituzionale e il funzionamento della Corte costituzionale
        sono regolati con legge costituzionale.
    \item Leggi ordinarie dello Stato (e atti aventi forza di legge: D.Lgs e D.L.). \newline
        Il potere di emanare le leggi è del Parlamento, che in Italia costituisce appunto l'organo 
        legislativo (il Governo è l'esecutivo, ossia amministra il Paese).\newline
        Per approvare una legge esiste un procedimento particolare. Un progetto di legge arriva 
        all'esame del Parlamento, che come sappiamo è composto dalla Camera dei deputati e dal Senato 
        della Repubblica. Ebbene, affinché la legge entri in vigore deve prima essere approvata, nella 
        stessa formulazione, dai due rami del Parlamento. Le due assemblee votano separatamente e, se 
        ci sono delle modifiche da deliberare, devono essere approvate da entrambe. In questo modo si 
        garantisce che il provvedimento sia la risultante di un dialogo ponderato tra maggioranza e 
        minoranza parlamentare, ossia tra tutti i deputati e senatori che rappresentano la popolazione italiana.
    \item Leggi regionali. \newline
        La Costituzione riconosce autonomia - anche legislativa - alle Regioni nelle materie indicate
        nell'Art. 117 o in altre materie eventualmente indicate con leggi costituzionali.\newline
        Nelle materie di competenza regionale, lo Stato può dettare con legge solo i principi
        fondamentali.\newline
        Le Regioni adottano la disciplina analitica.
    \item Regolamenti.\newline
        I Regolamenti sono emanati dal Governo o da altre autorità (es. Regioni, Province, Comuni, Banca d'Italia ecc.) e
        si suddividono in:
        \begin{itemize}
            \item Di esecuzione: Regolamenti che regolano nei particolari materia già disciplinate dalla
                legge.
            \item Indipendenti: Regolamenti che regolano materie non regolate da alcuna legge.
        \end{itemize}
    \item Usi. \newline
        Gli Usi o consuetudini sono una Fonte del diritto non scritta e non deliberata da soggetti 
        istituzionali (es. Parlamento, Regione, ecc.). Essi vengono definiti come una ripetizione
        generale, uniforme e costante di determinati comportamenti, con la libera convinzione di ottemperare
        a norme giuridicamente vincolanti.\newline
        Un'attuazione degli Usi in determinate materie è discriminata dalla regolamentazione della materia da parte 
        di una qualche legge. Gli usi hanno piena efficacia in materie \textbf{non} regolate da Leggi o Regolamenti.
\end{itemize}

\subsection{Differenza tra legge, decreto legge e decreto legislativo}

Dalla definizione di Legge riportata nel punto \emph{Leggi ordinarie dello Stato} delle Fonti 
del diritto è possibile discostarsi conoscendone la differenza con un decreto legge e un decreto
legislativo.
\begin{itemize}
    \item Decreto legge: Spesso l'approvazione di una legge può richiedere tempi molto lunghi, mesi
        o addirittura anni (abbiamo visto che deve essere approvata nella stessa formulazione da Camera
        e Senato). Spesso però ci sono delle situazioni di urgenza da fronteggiare, che richiedono un 
        provvedimento legislativo in tempi brevi. Ecco perché esiste il decreto legge: esso è un atto
        emanato dal Governo in casi di necessità e urgenza (si pensi ad una calamità naturale, ad esempio 
        un terremoto). Entro 60 giorni, però, il decreto legge deve essere convertito in legge dal Parlamento, 
        che è l'unico organo che ha potestà legislativa (si parla di «legge di conversione»). Se ciò non 
        avviene, il decreto legge decade con effetto retroattivo. In pratica, è come se non fosse mai esistito
        \begin{center}
            GOVERNO $\xrightarrow{POI}$ PARLAMENTO
        \end{center}
    \item Decreto legislativo: viene utilizzato quando le materie da disciplinare sono molto tecniche e il 
        Parlamento non ha le competenze necessarie per trattarle. Queste competenze si ritrovano invece nel Governo, 
        che è un organo amministrativo. \newline
        Di conseguenza, il Parlamento emette una legge delega diretta al Governo: in questo provvedimento vengono 
        stabiliti i criteri, i limiti (anche temporali) e l'oggetto del decreto legislativo da emanare. Il Governo poi 
        approva il decreto legislativo seguendo le disposizioni della legge delega
        In questo modo viene salvaguardata la separazione dei poteri: è il Parlamento a dettare tutti i criteri per l'emanazione
        del decreto, con il Governo che agisce nell'ambito della delega specifica
        \begin{center}
            PARLAMENTO $\xrightarrow{POI}$ GOVERNO
        \end{center}
\end{itemize}

\subsection{Regolamenti comunitari e direttive comunitarie}
I Regolamenti comunitari sono discipline dettagliate, direttamente applicabili negli Stati membri, appena
entrano in vigore devono essere rispettate all'interno dei singoli Stati. \newline
Le Direttive comunitarie dettano obiettivi, principi e regole generali. Gli Stati membri devono dare attuazione
alle direttive disciplinando la materia nei dettagli con propria legge interna, entro il termine stabilito dalla
direttiva stessa.
\subsection{Domanda su direttiva comunitaria...}
Se è stata pubblicata una direttiva ed è scaduto il termine per il recepimento nei singoli ordinamenti giuridici,
posso applicare direttamente la direttiva anche in assenza di una legge nazionale? \newline
Se uno Stato Membro non dà attuazione a una Direttiva europea entro il termine prefissato e se la Direttiva contiene
disposizioni sufficientemente precise, essa sarà direttamente applicabile.

\subsection{Efficacia della legge nel tempo}
In quale momento una legge inizia ad avere efficacia? E quando termina di averne? \newline
Un regolamento o una legge entra in vigore il 15º giorno successivo alla pubblicazione di quest ultimo/a sulla Gazzetta Ufficiale. \newline
La legge perdura fino ad:
\begin{itemize}
    \item\emph{abrogazione espressa}: abrogazione espressa da una disposizione di legge successiva o referendum popolare o sentenza
        di illegittimità costituzionale 
    \item\emph{abrogazione tacita}: abrogazione implicita dovuta ad una nuova disposizione di legge o ad una legge che ne regola 
        l'intera materia
\end{itemize}

\subsection{Efficacia della legge nello spazio}
Secondo la \emph{Nazionalità} del diritto ogni Stato formula a propria discrezione le proprie norme
di diritto. Tuttavia non è detto che sul territorio di uno Stato si applichi solo il diritto dello Stato stesso. \newline
I possibili conflitti che possono nascere nell'applicazione di norme di stati diversi vanno trattati 
nell'ambito del \emph{diritto internazionale privato} e delle \emph{Convenzioni internazionali}.

\subsection{Quali sono i principali criteri di interpretazione della legge?}
Applicare la legge non è un procedimento meccanico: una volta individuata la legge da applicare al caso che 
ci interessa, dobbiamo attribuirle un significato, ossia interpretarla.\newline
L'interpretazione della legge segue diversi criteri:
\begin{itemize}
    \item \emph{Letterale}: dal significato proprio delle parole secondo la loro connessione
    \item \emph{Teleologica}: dall'interpretazione del legislatore che può essere \emph{estensiva} o \emph{restrittiva} nei
        confronti del senso letterale
    \item \emph{Analogia}: se manca una regola specifica per risolvere una controversia si fa riferimento alle disposizioni che
        regolano casi simili o materie analoghe [questo criterio non è applicabile per le norme penali e per 
        le norme eccezionali]
\end{itemize}
In qualunque altro caso si fornisce alla legge un'interpretazione secondo i principi generali
dell'ordinamento giuridico dello Stato. Questi principi possono essere ottenuti per induzione da un insieme di norme.
(es. libera circolazione della ricchezza, conservazione del contratto ecc.).

\subsection{Gradi di giudizio}
Con grado di giudizio si indica la fase delimitata da una sentenza o da una qualunque decisione effettuata da un
organo giudicante.
\begin{itemize}
    \item Giudice di primo grado: il giudice esamina per la prima volta la causa
    \item Giudice di secondo grado: il giudice riesamina e si pronuncia sulla stessa causa per la
        seconda volta (Corte d'Appello)
    \item  Giudizio di legittimità: viene effettuato un controllo sulla legalità dei precedenti gradi
        di giudizio senza affrontare di nuovo la controversia nel merito (Corte di Cassazione)
\end{itemize}

\subsection{Persona fisica e persona giuridica}
Definiamo la Persona come il soggetto di diritto, centro di imputazione di rapporti giuridici (titolare di diritti e di doveri).\newline
Una \emph{Persona fisica} è un singolo individuo, mentre una \emph{Persona giuridica} è un'organizzazione collettiva che compie atti giuridici
attraverso i suoi organi (es. S.r.l., S.p.A.).

\subsection{Capacità giuridica e capacità di agire}
Definiamo la capacità giuridica come l'attitudine ad essere titolari di diritti e di doveri. Essa si acquista alla nascita e dura fino alla morte.\newline
Definiamo la capacità di agire come l'attitudine a compiere atti giuridici con i quali acquistare diritti o assumere doveri. Essa si acquista con la maggiore età.

\subsection{Diritti reali}
Per diritti reali intendiamo delle facoltà che il titolare può esercitare sulle cose. Essi sono diritti assoluti, cioè spettano nei confronti di tutti i
soggetti. (Esempio: proprietà, superficie, usufrutto, ecc...)

\subsection{Obbligazioni e fonti delle obbligazioni}
L'obbligazione è un rapporto giuridico in forza del quale un soggetto, detto debitore, è tenuto a una determinata prestazione, suscettibile 
di valutazione economica, a favore di un altro soggetto, detto creditore.
Le fonti delle obbligazioni sono:
\begin{itemize}
    \item \textbf{Contratto}: accordo di due o più parti per costruire, regolare o estinguere tra loro un rapporto giuridico patrimoniale.
    \item \textbf{Fatto illecito}: ogni fatto doloso o colposo che cagiona ad altri un danno ingiusto obbliga al risarcimento del danno.
\end{itemize}

\subsection{Contratto}
Il contratto è un accordo di due o più parti per costituire, regolare o estinguere tra loro un rapporto giuridico patrimoniale.\newline
I requisiti del contratto sono:
\begin{itemize}
    \item Accordo delle parti: il contratto si intende concluso quando il proponente riceve notizia della accettazione
    \item Causa: funzione economico-sociale del contratto (esempio: vendita)
    \item Oggetto: diritto che il contratto trasferisce (o prestazione che la parte si obbliga a eseguire). Deve essere possibile, lecito, determinato
        o determinabile.
    \item Forma: modalità con cui la volontà contrattuale può essere manifestata. Di norma, essa è libera, 
        ossia non è richiesto per la validità del contratto che essa sia manifestata in un modo particolare. Anche un accordo verbale od un comportamento 
        concludente possono quindi fare sorgere obbligazioni contrattuali.
        Devono farsi per atto pubblico o per scrittura privata, sotto pena di nullità, tra gli altri, i contratti che trasferiscono la proprietà di beni 
        immobili o che costituiscono, modificano o trasferiscono diritti reali su beni immobili.\newline
        Nel caso di un contratto stipulato in una forma non iscritta può portare a dissidi sull'oggetto delle prestazioni, sulla loro qualità o sul corrispettivo: è 
        infatti più difficile dimostrare, in caso di controversia davanti al giudice, quali erano gli accordi iniziali. \newline

        Nel caso di un articolo di contratto che violi una legge nazionale o comunque locale il contratto rimane valido ma il proponente è soggetto
        al pagamento di una cospicua sanzione stabilita dal legislatore.

\end{itemize}

\subsection{Nullità e annullabilità del contratto}
Un contratto si dice nullo se:
\begin{itemize}
    \item è contrario a norme imperative
    \item manca un requisito essenziale
    \item è illecito nella causa o nei motivi
    \item l'oggetto non ha i requisiti previsti
    \item ci sono casi stabiliti dalla legge che lo rendono nullo
\end{itemize}
La nullità di un contratto può essere fatta valere da chiunque, può essere rilevata d'ufficio dal giudice e non si prescrive.
\newline
Un contratto è annullabile se una parte era incapace di contrattare, o se il consenso è stato ottenuto per errore, violenza o dolo.\newline
L'annullamento può essere richiesto solo dalla parte nel cui interesse è previsto e si prescrive in 5 anni.
\subsection{Contratti tipici e atipici}
Definiamo contratti tipici quei contratti che hanno una \emph{causa} tipica, cioè prevista e regolamentata dalla legge (es. vendita).\newline
I contratti atipici, invece, non hanno una causa tipica. Di volta in volta occorre accertare la ricorrenza o meno di una causa 
(intesa come funzione economico-sociale. Esempio la licenza di un software)

\subsection{Responsabilità}
L'ordinamento giuridico italiano distingue tra tre tipi di responsabilità civile:
\begin{itemize}
    \item Responsabilità contrattuale: deriva da un contratto. Sorge quando sono violate una o più clausole del contratto. 
        (es. un ingegnere non consegna il software alla scadenza pattuita e deve pagare una penale)
    \item Responsabilità extracontrattuale: deriva da un fatto illecito; si causa un danno ingiusto con un comportamento doloso o colposo 
        (es. si rompe un oggetto altrui per distrazione)
    \item Responsabilità indiretta: responsabilità derivante dalle azioni di un altro soggetto 
        (responsabilità di un genitore su fatti illeciti commessi dai figli)
    \item Responsabilità oggettiva: si risponde di un fatto anche se lo si è commesso senza dolo e senza colpa, per il semplice
        rapporto di causalità tra attività e danno. Ci si libera con la prova di aver fatto tutto il possibile per evitare il danno. 
        (es. esercizio di attività pericolose, animali in custodia ecc.)
\end{itemize}

\subsection{Domande sulla stipulazione di un contratto}
\textbf{Il contratto, in generale, deve essere stipulato per iscritto e firmato?
E' valido un contratto verbale? Quali problemi potrebbero sorgere da un contratto non scritto?}
Il fatto che un contratto sia scritto non rappresenta una condizione
inderogabile o tassativa, può essere considerato allo stesso modo valido anche
se dovesse essere concluso a voce o in forma elettronica.
Il fatto che non sia scritto, non rappresenta una condizione che fa venire meno la
sua natura di "contratto", ma senza una prova, stabile nel tempo, del contenuto
delle intese raggiunte o dei rispettivi obblighi, potrebbero nascere possibili
contestazioni e il ricorso a testimoni.
