\newpage
\section{Il commercio elettronico}
\subsection{Che cosa sono i servizi della società dell’informazione (SSI)?}
I servizi elettronici della società dell'informazione sono attività economiche svolte online.
I servizi vengono prestati normalmente dietro retribuzione nei seguenzi modi:
\begin{itemize}
    \item a distanza: servizio fornito senza la presenza simultanea delle parti
    \item per via elettronica: servizio inviato all'origine e ricevuto a destinazione grazie a mezzi elettronici e digitali, come la compressione digitale e la memorizzazione di dati.
\end{itemize}
Il servizio viene effettuato e ricevuto mediante fili, radio, mezzi ottici o mezzi elettromagnetici.
\begin{itemize}
    \item a richiesta individuale di un destinatario di servizi, il servizio viene fornito mediante tramissione di dati.
\end{itemize}

\subsection{A quale obbligo informativo generale è soggetto chi fornisce beni e servizi via Internet
(prestatore di SSI)?}

Il prestatore deve fornire le seguenti informazioni ai destinatari di SSI e alle autorità competenti:
\begin{itemize}
    \item nome, denominazione o ragione sociale;
    \item domicilio o sede legale
    \item dati per essere contattato (inclusa email)
    \item numero di iscrizione al registro delle imprese
    \item estremi di autorità di vigilanza
    \item dati ulteriori per professioni regolamentate
    \item numero della partita IVA
    \item prezzi e tariffe (indicando imposte e spese di spedizione)
    \item informazioni ulteriori per attività soggette ad autorizzazione
\end{itemize}

\subsection{Informazioni obbligatorie per comunicazioni commerciali.}

Le comunicazioni commerciali contengono informazioni chiare da cui l'utente apprende:
\begin{itemize}
    \item che si tratta di una comunicazione commerciale
    \item il soggetto (persona fisica o giuridica) per conto della quale la comunicazione è effettuata
    \item che si tratta di un'offerta promozionale e ne apprende le relative condizioni d'accesso
    \item che si tratta di concorsi o giochi promozionali
\end{itemize}

\subsection{Informazioni per la conclusione del contratto.}
Nei contratti con un consumatore, il prestatore di SSI, prima dell'inoltro dell'ordine deve fornire informazioni su:
\begin{itemize}
    \item fasi tecniche per la conclusione del contratto
    \item modalità di archiviazione e accesso al contratto concluso
    \item mezzi tecnici a disposizione dell'utente per correggere errori di inserimento dei dati prima di inoltrare l'ordine
    \item Eventuali codici di condotta a cui il prestatore aderisce
    \item lingue per concludere il contratto, oltre all'italiano
    \item strumenti per la composizione delle controversie
\end{itemize}
Inoltre le clausole e le condizioni generali del contratto devono essere messe a disposizione dell'utente per la memorizzazione e riproduzione.
\newline
Rimangono validi gli obblighi relativi previsti su specifici beni e servizi, e quelli previsti sui contratti a distanza
\newline
Le norme non si applicano:
\begin{itemize}
    \item ai contratti in cui le parti non sono consumatori
    \item ai contratti conclusi esclusivamente via posta elettronica
\end{itemize}

\subsection{Comunicazioni commerciali non sollecitate. Che cosa dice il decreto? Come si collega con la
disciplina sulla privacy?}

Le comunicazioni commerciali non sollecitate inviate per posta elettronica devono:
\begin{itemize}
    \item essere identificate come tali nel momento della ricezione
    \item contenere indicazioni sull'opposizione del destinatario in futuro al ricevimento di tali comunicazioni.
\end{itemize}
È compito del prestatore di SSI provare che le comunicazioni commmerciali siano state sollecitate.
\newline
Il Garante per la protezione dei dati personali ha quindi ripetutamente affermato che l’invio di e-mail pubblicitarie senza il consenso del destinatario effettuato a fini di profitto viola anche una norma penale.
\newline
Tutte le comunicazioni commerciali inviate via mail \textbf{devono essere preventivamente autorizzate dal destinatario}.

\subsection{Come avviene l’inoltro dell’ordine? La ricevuta del prestatore è necessaria affinché il contratto on-line possa considerarsi concluso? (Risposta: NO)}
L'inoltro dell'ordine avviene applicando le norme generali di conclusione dei contratti.
Il prestatore deve fornire al destinatario la ricevuta dell'ordine, con un riepilogo delle condizioni del contratto, informazioni sul bene o servizio, prezzo, mezzi di pagamento, recesso.
\newline
Ordine e ricevuta vengono considerati pervenuti quando le parti hanno possibilità di accedervi.
Queste disposizioni non si applicano ai contratti conclusi esclusivamente mediante posta elettronica.

\subsection{Quando è concluso un contratto on line?}
La conclusione del contratto telematico (o anche “contratto online”) può? avvenire generalmente mediante tre modalità?:
\begin{itemize}
    \item l’accettazione dell’utente sul sito internet del proponente: il contratto telematico si conclude, dopo essersi registrati e aver compilato un form (modello) elettronico, mediante il meccanismo del “point and click”, cioè facendo un click sul tasto “accettazione”. Questa è una modalità di conclusione del contratto online a cui il consumatore deve prestare particolarmente attenzione, poiché l’automatismo che vi è implicato può condurre alla conclusione del contratto telematico senza la dovuta attenzione, se non, in alcuni casi, anche alla conclusione del contratto involontaria (selezione del tasto di “accettazione” per errore, ad esempio).
    \item L’accettazione del venditore (merchant):  essa si verifica quando sul sito del venditore vi sono le indicazioni “senza impegno” o “salvo conferma”. Il contratto online si considererà pertanto concluso solo quando il venditore, dopo un’eventuale trattativa con il potenziale acquirente, accetti la proposta ricevuta. Tale tecnica è solitamente impiegata per consentire alle aziende di gestire le domande ed ovviare alle problematiche di approvvigionamento dei magazzini.
    \item L’accettazione tramite scambio di e-mail: esso si basa sullo scambio di corrispondenza e-mail tra le parti (non di posta certificata): il proponente invia un’e-mail contenente una proposta alla quale il destinatario risponde mediante l’invio con un’e-mail di accettazione. Il contratto online si considera concluso nel momento in cui l’e-mail contenente l’accettazione della proposta contrattuale giunge all’indirizzo del proponente o presso il server del fornitore di posta elettronica del proponente.
\end{itemize}
Una volta concluso il contratto online, l’impresa deve, senza ritardo e per via telematica, inviare ricevuta dell'ordine del destinatario contenente le seguenti informazioni:
\begin{itemize}
    \item un riepilogo delle condizioni generali e particolari applicabili al contratto concluso;
    \item le informazioni relative alle caratteristiche essenziali del bene o del servizio oggetto del contratto telematico;
    \item l'indicazione dettagliata del prezzo e dei mezzi di pagamento;
    \item l'indicazione dettagliata del recesso, dei costi di consegna e dei tributi applicabili.
\end{itemize}
L'ordine e la ricevuta si considerano pervenuti quando le parti, alle quali sono indirizzati, hanno la possibilità di accedervi. In altri termini, il venditore non deve dimostrare che il destinatario della comunicazione abbia anche letto il relativo contenuto. Una volta ricevuta la comunicazione, il relativo contenuto si presume conosciuto (o, comunque, conoscibile). Quindi, se il destinatario della comunicazione omette di esaminarne il contenuto, non potrà – salvo situazioni eccezionali – sostenere di non esserne a conoscenza.
\newline
Nei contratti conclusi online qualora il contratto contenga una o più clausole vessatorie, esse dovranno essere specificatamente approvate, con le seguenti modalità:
\begin{itemize}
    \item mediante l’invio della versione cartacea del contratto contenente la doppia firma;
    \item mediante firma digitale o elettronica (modalità tuttavia non molto diffusa e poco pratica);
    \item mediante un secondo “point and click”, oltre a quello di adesione al contratto.
\end{itemize}
Qualora il contratto online sia concluso tra un professionista (persona fisica o giuridica, pubblica o privata, che stipula il contratto nel contesto della propria attività imprenditoriale o professionale) ed un consumatore (ossia persona fisica che contrae per scopi estranei all’attività imprenditoriale o professionale eventualmente svolta), si applica anche la disciplina del Codice del Consumo. Cioè quell’insieme di norme poste a tutela della parte contrattuale considerata “debole” – cioè il consumatore – in quanto agisce al di fuori della propria attività professionale (dove di regola si presume una maggior perizia e consapevolezza nel condurre gli affari).
\newline
Il Codice del Consumo prevede un elenco di ulteriori clausole – eventualmente inserite in un contratto - che si presumono vessatorie fino a prova contraria. Ciò poiché, malgrado la buona fede, determinano a carico del consumatore un significativo squilibrio dei diritti e degli obblighi derivanti dal contratto. Esse pertanto dovranno essere specificatamente approvate dal consumatore contraente. In mancanza di specifica approvazione, il contratto online può essere impugnato.

\subsection{Occorre la firma elettronica per concludere un contratto on-line? Sì/no? Perché?}
No, poiché dipende dalla tipologia di contratto.
Alcuni contratti prevedono soltanto uno scambio di mail, altri mediante point and click.

\subsection{Responsabilità dei prestatori di servizi di mere conduit, caching, hosting. Questi prestatori
hanno un obbligo generale di sorveglianza? Quali obblighi hanno?}
Esistono diverse responsabilità dei prestatori in base alla tipologia di servizio offerto:
\begin{itemize}
    \item mere conduit: trasmissione di informazioni fornite da un destinatario di un servizio o di fornitura di accesso.
    Il prestatore non risulta responsabile se:
    \begin{itemize}
        \item Non dà origine alla trasmissione
        \item Non seleziona il destinatario
        \item Non seleziona o modifica le informazioni trasmesse
    \end{itemize}
    \item Caching: trasmissione di informazioni fornite da un destinatario, con memorizzazione automatica, intermedia e temporanea.
    Il prestatore non è responsabile se:
    \begin{itemize}
        \item Non modifica le informazioni
        \item Si conforma alle condizioni di accesso e di aggiornamento delle informazioni
        \item Non interferisce con uso di tecnologia lecita per ottenere dati sull'impiego delle informazioni
        \item Agisce prontamente per rimuovere informazioni memorizzate o disabilita l'accesso se viene effettivamente a conoscenza di operazioni potenzialmente illecite.
    \end{itemize}
   \item Hosting: memorizzazione di informazioni fornite da un destinatario del servizio.
   Un prestatore non è responsabile se:
   \begin{itemize}
       \item non è effettivamente a conoscenza di illeciti
       \item rimuove o disabilita l'accesso a informazioni appena viene a conoscenza di illeciti.
   \end{itemize}
\end{itemize}
In tutte e tre le tipologie di servizi l'autorità giudiziaria o amministrativa può esigere anche in via d'urgenza, che il prestatore impedisca o ponga fine alle violazioni.
\newline
I prestatori dei servizi sopra elencati:
\begin{itemize}
    \item non hanno un obbligo generale di sorveglianza sulle informazioni che trasmettono o memorizzano
    \item non hanno un obbligo generale di ricerare attivamente fatti o circostanze che indichino la presenza di attività illecite
    \item devono informare le autorità competenti se vengono a conoscenza di attività potenzialmente illecite.
    \item Devono fornire alle autorità competenti informazioni per identificare i destinatari al fine di individuare e prevenire attività illecite.
\end{itemize}

\subsection{Se Lei dovesse creare un sito web di commercio elettronico, quali problemi giuridici si porrebbe?}

\subsection{Che cos’è un contratto a distanza secondo il Codice del consumo?}
Un contratto a distanza:
\begin{itemize}
    \item ha per oggetto beni o servizi
    \item è stipulato mediante accordo tra professionista e un consumatore nell'ambito di prestazione di servizi a distanza o di regime organizzato di vendita
    \item per il contratto si utilizzano tecniche di comunicazione a distanza, conclusione del contratto stesso inclusa.
\end{itemize}

\subsection{Quali sono le principali informazioni obbligatorie che devono essere fornite al consumatore
nell’ambito di un contratto a distanza?}
Le informazioni obbligatorie previste in un contratto a distanza sono:
\begin{itemize}
    \item caratteristiche principali del bene o del servizio
    \item identità del professionista
    \item indirizzo geografico del professionista, email, telefono, fax
    \item prezzo totale del bene o del servizio (incluse imposte e spese di spedizione)
    \item costo di utilizzo del mezzo di comunicazione a distanza, se diverso dalla tariffa base
    \item modalità di pagamento e consegna, data, consegna gestione reclami
    \item diritto di recesso (o eventuale esclusione)
    \item modalità, tempi e costi di restituzione del bene
    \item promemoria della garanzia legale di conformità
    \item condizioni di assistenza e garanzie commerciali
    \item durata della validità dell'offerta e del prezzo
    \item durata minima del contratto, modalità di recedere (arretrare)
    \item durata minima degli obblighi del consumatore a norma del contratto
    \item l'esistenza e le condizioni di depositi o altre garanzie finanziarie che il consumatore è tenuto a pagare o fornire su richiesta del professionista
    \item la funzionalità del contenuto digitale, comprese le misure applicabili di protezione tecnica
    \item qualsiasi interoperabilità del contenuto digitale con hardware e software
    \item la possibilità di servirsi di un meccanismo extra giudiziale di reclamo e ricorso e le condizioni per avervi accesso
\end{itemize}

\subsection{Conferma scritta.}
La conferma del contratto concluso avviene su un mezzo durevole ed entro un termine ragionevole dopo la conclusione del contratto a distanza e al più tardi al momento della consegna dei beni, oppure prima che l'esecuzione del servizio abbia inizio.
\newline
La conferma scritta comprende:
\begin{itemize}
    \item tutte le informazioni obbligatorie, a meno che il professionista non abbia già fornito l'informazione al consumatore su un mezzo durevole prima della conclusione del contratto a distanza
    \item la conferma del previo consenso espresso e dell'accettazione del consumatore in caso di fornitura di contenuto digitale mediante un supporto non materiale
\end{itemize}

\subsection{Esecuzione del contratto.}
\subsection{Il diritto di recesso. Che cos’è, come si esercita, quali effetti ha. (BENE!!)
Può farmi qualche esempio di casi in cui non può essere esercitato il diritto di recesso?}

Il diritto di recesso pone termine tra consumatore e professionista:
\begin{itemize}
    \item di eseguire il contratto a distanza
    \item di concludere un contratto a distanza nel caso in cui un'offerta sia stata fatta dal consumatore.
\end{itemize}
Nel caso di un acquisto online il diritto di recesso consente di avere un rimborso, senza penalità e senza dover dichiarare un motivo:
\begin{itemize}
    \item la fornitura di beni che rischiano di deteriorarsi o scadere rapidamente (ad esempio alimentari);
    \item nel caso in cui si sia comprato un bene, entro 14 giorni dalla ricezione fisica del bene
    \item nel caso di un servizio entro 14 giorni dalla conclusione del contratto
\end{itemize}
Nel caso in cui un professionista non fornisce alcuna informazione sul diritto di recesso, viene esteso il periodo di recesso: esso termina dodici mesi dopo la fine del periodo di recesso iniziali (di durata di 14 giorni)

Nel caso in cui il professionista fornisce al consumatore le informazioni sul diritto di recesso entro dodici mesi dalla data del periodo di recesso iniziale, il periodo di recesso termina 14 giorni dopo il giorno in cui il consumatore riceve le informazioni.


Prima della scadenza del periodo di recesso, il consumatore informa il professionista della sua decisione di esercitare il diritto di recesso dal contratto, utilizzando una delle seguenti modalità:
\begin{itemize}
    \item Un modulo di recesso fornito dal professionista
    \item presentando una qualsiasi altra dichiarazione esplicita della sua decisione di recedere dal contratto.
\end{itemize}
Il consumatore esercita correttamente il diritto al recesso solo se invia una dichiarazione prima della scadenza del periodo di recesso.
Solo il consumatore può provare l'esercizio del diritto di recesso.
\newline
Esempi di eccezioni al diritto di recesso sono:
\begin{enumerate}
    \item contratti di servizi dopo la completa prestazione del servizio se l'esecuzione è iniziata con l'accordo espresso del consumatore e con l'accettazione della perdita del diritto di recesso a seguito della piena esecuzione del contratto da parte del professionista;
    \item la fornitura di beni o servizi il cui prezzo è legato a fluttuazioni nel mercato finanziario;
    \item la fornitura di beni confezionati su misura o chiaramente personalizzati (ad esempio braccialetti con nome);
    \item la fornitura di beni sigillati che non si prestano ad essere restituiti per motivi igienici o connessi alla protezione della salute e sono stati aperti dopo la consegna (ad esempio siringhe);
    \item la fornitura di beni che, dopo la consegna, risultano, per loro natura, inscindibilmente mescolati con altri beni;
    \item la fornitura di bevande alcoliche, il cui prezzo sia stato concordato al momento del contratto, la cui consegna possa avvenire solo dopo 30 giorni e il cui valore effettivo dipenda da fluttuazioni sul mercato;
    \item i contratti in cui il consumatore ha specificamente richiesto una visita da parte del professionista ai fini dell'effettuazione di lavori urgenti di riparazione o manutenzione;
    \item la fornitura di registrazioni audio o video sigillate o di software sigillati che sono stati aperti dopo la consegna;
    \item la fornitura di giornali, periodici e riviste ad eccezione dei contratti di abbonamento;
    \item i contratti conclusi in occasione di un'asta pubblica;
    \item la fornitura di alloggi per fini non residenziali, il trasporto di beni, i servizi di noleggio auto, catering o tempo libero se il contratto prevede una data o un periodo di esecuzione specifici;
    \item la fornitura di contenuto digitale mediante un supporto non materiale se l'esecuzione è iniziata con l'accordo espresso del consumatore e con la sua accettazione del fatto che in tal caso avrebbe perso il diritto di recesso.
\end{enumerate}

\subsection{Quali sono gli obblighi del consumatore e del professionista in caso di esercizio del diritto di
recesso?}
Gli obblighi del professionista, in caso di recesso, sono:
\begin{itemize}
    \item rimborso di tutti i pagamenti ricevuti dal consumatore, incluse le spese di consegna, entro i 14 giorni della decisione del consumatore sulla recisione del contratto.
\end{itemize}
Il rimborso avviene con lo stesso mezzo di pagamento usato dal consumatore per la transazione iniziale.
È nulla qualsiasi clausola che prevede limitazioni delle somme versate in conseguenza dell'esercizio del diritto di recesso.
Inoltre il professionista non è tenuto a rimborsare i costi supplementari, qualora il consumatore abbia scelto espressamente un tipo di consegna diversa dal tipo meno costoso di consegna offerto dal professionista.
Il professionista può trattenere il rimborso finché non abbia ricevuto i beni o finché il consumatore non abbia dimostrato di avere rispedito i beni.
\newline
\newline
Gli obblighi del consumatore in caso di recesso sono:
\begin{enumerate}
    \item dovere rispedire i beni entro i 14 giorni dalla data in cui ha deciso di richiedere il rimborso.
    \item sostenere soltanto il costo diretto della restituzione dei beni, purché il professista non abbia concordato di sostenerlo, o abbia omesso di informare il consumatore che tale costo spetta al consumatore.
    \item responsabilità della diminuzione del valore dei beni risultante da una manipolazione dei beni diversa da quella necessaria per provare il funzionamento e l'integrità del bene stesso.
\end{enumerate}
Il consumatore non risulta responsabile per la diminuzionedel valore dei beni, se il professionista non ha informato il consumatore sul diritto di recesso

\subsection{Qual è il foro competente in caso di controversia tra un professionista e un consumatore?
Perché?}
Il foro competente in caso di controversia tra un professionista e un consumatore è il giudice del luogo di residenza o di domicilio del consumatore, se si trovano nel territorio dello stato.
