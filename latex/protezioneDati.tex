\newpage
\section{Protezione dei dati}

\subsection{Chi ha diritto alla protezione dei dati personali?}
Qualsiasi persona fisica a cui appartengono i dati personali in questione.

\subsection{A quali trattamenti di dati si applica il Regolamento sulla protezione dei dati personali?
Ambito di applicazione territoriale – ambito di applicazione materiale}

Il regolamento sulla protezione dei dati personali si applica su qualsiasi dato personale da cui è possibile risalire a una persona fisica; non viene applicato sui dati anonimi, da cui sia impossibile dedurre la persona a cui appartengono.
\newline
L'ambito di \textbf{applicazione materiale} prevede il regolamento dei dati personali in caso di trattamento di dati:
\begin{itemize}
    \item automatizzato interamente o parzialmente
    \item non automatizzato; i dati sono contenuti in un archivio o destinati in un archivio
\end{itemize}
Il regolamento non viene applicato nel caso in cui il trattamento di dati personali è effettuato da:
\begin{itemize}
    \item \textit{Persone fisiche} per l'esercizio di attività esclusivamente personali o domestiche
    \item \textit{Autorità competenti}, ai fini di prevenzioni, indagine o accertamento o perseguimento di reati, o salvaguardia della salute pubblica
\end{itemize}
L'ambito di \textbf{applicazione territoriale} prevede che il trattamento dei dati personali sia effettuato da:
\begin{itemize}
    \item Titolare o responsabile stabilito in UE
    \item Titolare o responsabile non stabilito in UE, ma tratta dati di interessati che si trovano in UE per:
        \begin{itemize}
            \item  prestazioni di servizi o offerta di beni
            \item monitorare il loro comportamento all'interno dell'UE
        \end{itemize}
\end{itemize}

\subsection{Se tratto dati personali per scopi esclusivamente personali sono soggetto al Regolamento?}
No, in base a quanto suggerito dall'ambito di \textit{applicazione materiale}.


\subsection{Che cos’è un dato personale?}
Un dato personale è qualsiasi informazione che riguarda una persona fisica identificata o identificabile;
quest'ultima può essere identificata direttamente o indirettamente, nei seguenti modi:
\begin{itemize}
    \item con un identificativo come il nome, un numero di identificazione
    \item con dati relativi all'ubicazione (dati relativi a immobili)
    \item con un identificativo online a uno o a più elementi caratteristici della sua identità fisica, fisiologica, genetica, psichica, economica, culturale, sociale
\end{itemize}
Esiste una distinzione tra dati:
\begin{itemize}
    \item comuni
    \item sensibili
\end{itemize}

\subsection{Che cosa si intende per trattamento? Esempi di attività che costituiscono un trattamento di
dati personali.}
Per \textit{trattamento} si intende qualsiasi operazione o insieme di operazioni, compiute con o senza l'ausilio di processi automatizzati e applicate a dati personali o insiemi di dati personali come:
\begin{itemize}
    \item raccolta
    \item registrazione
    \item organizzazione
    \item strutturazione
    \item conservazione
    \item adattamento o modifica
    \item estrazione
    \item consultazione
    \item uso
    \item diffusione o qualsiasi altra forma di messa a disposizione
    \item raffronto o interconnnessione
    \item limitazione
    \item cancellazione o distruzione
\end{itemize}

\subsection{Se memorizzate il cellulare di un vostro compagno di corso che incontrate in giro dovete fornirgli
l’informativa privacy? Si/no? Perché?}
No, in quanto l'utilizzo del suo numero di telefono è esclusivamente personale.

\subsection{L'indirizzo IP/un nickname sono dati personali? Perché?}
L'indirizzo IP o un nickname sono dati personali, poiché da essi è possibile risalire alla persona da cui derivano, e occorre un trattamento \textit{ad hoc} in base alle circostanze di utilizzo

\subsection{Il Regolamento si applica ai dati anonimi? Quand’è che un dato è anonimo?}
Il regolamento non si applica ai dati anonimi, poiché non è possibile risalre a una persona da essi.
Un dato risulta anonimo quando è impossibile ricostruire l'identità della persona a partire da quel dato.

\subsection{Cosa significa anonimizzare e pseudonimizzare un dato?}
Per \textit{pseudonimizzazione} si intende che il trattamento dei dati personali avviene in modo tale che i dati personali non possano più essere attribuiti a una persona specifica, senza l'utilizzo di informazioni aggiuntive. Queste ultime devono essere conservate separatamente e devono essere soggette a misure tali da garantire che tali dati personali non siano attribuiti a una persona fisica identificata o identificabile.

\subsection{Chi sono i soggetti coinvolti nel trattamento?}
I ruoli coinvolti nel trattamento dei dati personali sono i seguenti:
\begin{itemize}
    \item \textbf{Titolare}: persona fisica o giuridica, autorità pubblica, o servizio o organismo che, singolarmente o insieme ad altri, determina le finalità e i mezzi dei dati personali.
    Il titolare tratta dati per conto proprio.
    \newline
    Esempio: una società che tratta i dati dei propri clienti e dipendenti.

    \item \textbf{Responsabile}: la persona fisica o giuridica, autorità pubblica, il servizio o altro organismo che tratta dati personali \textit{per conto} del titolare del trattamento.
    \newline
    Esempio: consulente del lavoro che tratta dati dei dipendenti delle società per cui elabora buste paga
    \newline
    Esempio 2: fornitore cloud che tratta dati per conto dei propri clienti
    \newline
    Esempio 3: call center che contatta i clienti dei propri committenti

    \item \textbf{Interessato}: persona fisica a cui si riferiscono i dati personali
    Esempio: utente, cliente

    \item \textbf{Soggetto designato}: persone fisiche che operano sotto l'autorità del Titolare e del Responsabile e che svolgono specifici compiti e funzioni connessi al trattamento di dati personali. Nominati con atti di delega

    \item \textbf{DPO}: ha il compito di verificare l'applicazione del regolamento e le politiche di protezione dei dati; informa e fornisce consulenza al titolare sugli obblighi del regolamento e altre norme.
    Se richiesto, dà un parere in merito alla valutazione d'impatto sulla protezione dei dati e ne sorveglia lo svolgimento.
\end{itemize}

\subsection{Come si fa a nominare un responsabile del trattamento? Posso farlo a voce?}
No, per nominare un responsabile del trattamento occorre un contratto o un atto giuridico che vincoli il responsabile al titolare e che stipuli:
\begin{itemize}
    \item la materia disciplinata
    \item la durata del trattamento
    \item la natura del trattamento
    \item la finalità del trattamento
    \item il tipo di dati personali
    \item le categorie di interessati
    \item gli obblighi e i diritti del titolare del trattamento.
\end{itemize}

\subsection{Secondo quali principi fondamentali deve essere effettuato un trattamento?}

I principi applicabili al trattamento dei dati personali sono:
\begin{enumerate}
    \item \textbf{Liceità, correttezza, trasparenza}: i dati devono essere trattati in modo lecito, corretto e trasparente nei confronti della persona coinvolta
    \item \textbf{Limitazione della finalità}: i dati devono essere raccolti per finalità esplicite, legittime e in modo che le finalità rimangano le stesse; i trattamenti con fini:
    \begin{itemize}
        \item di ricerca scientifica
        \item di ricerca storica
        \item di archiviazione nel pubblico interesse
        \item statistici
    \end{itemize}
    non risultano incompatibili con le finalità iniziali.

    \item \textbf{Minimizzazione dei dati}: i dati devono essere adeguati, pertinenti e limitati a quanto necessario rispetto alle finalità per cui vengono trattati

    \item \textbf{Esattezza}: i dati devono essere esatti ed eventualmente aggiornati.
    In caso di dati inesatti essi devono essere tempestivamente rettificati o cancellati.

    \item \textbf{Limitazione della conservazione}: i dati devono essere conservati in una forma che consenta l'identificazione degli interessati per un arco di tempo non superiore al conseguimento delle finalità specificate; i dati possono essere conservati per periodi più lunghi, a patto che siano trattati a fini:
    \begin{itemize}
        \item di ricerca scientifica
        \item di ricerca storica
        \item di archiviazione nel pubblico interesse
        \item statistici
    \end{itemize}

    \item \textbf{Integrità e riservatezza}: i dati devono essere trattati in modo da garantire un'adeguata sicurezza, compresa la protezione di essi da trattamenti non autorizzati o illeciti o dalla perdita, distruzione o danni accidentali.

    \item \textbf{Responsabilizzazione}: il titolare è competente per il rispetto di questi principi e deve essere in grado di comprovarlo e dimostrarlo.
\end{enumerate}

\subsection{Liceità del trattamento}
Il trattamento risulta lecito solo se è avviene almeno una delle seguenti condizioni:
\begin{enumerate}
    \item L'interessato ha dato il suo consenso al trattamento dei propri dati personali per una o più specifiche finalità
    \item Il trattamento è necessario all'esecuzione di un contratto di cui l'interessato è parte o all'esecuzione di misure precontrattuali adottate su richiesta dello stesso

    \item Il trattamento è necessario per adempiere un obbligo legale al quale è soggetto il titolare del trattamento

    \item Il trattamento è necessario per la salvaguardia degli interessi vitali dell'interessato o di un'altra persona fisica

    \item Il trattamento è necessario per un compito di pubblico interesse o connesso all'esercizio di poteri pubblici in cui è coinvolto il titolare del trattamento

    \item Il trattamento è necessario per il perseguimento del legittimo interesse del titolare del trattamento o di terzi, a condizione che non prevalgano gli interessi o i diritti e le libertà fondamentali dell'interessato che richiedono la protezione dei dati personali, in particolare se l'interessato è un minore.

\end{enumerate}
NB: la regola non viene applicata al trattamento dei dati effettuato dalle autorità pubbliche nell'esecuzione dei loro compiti.

\subsection{Trattamento di categorie particolari di dati personali}

Vengono definiti dati sensibili:
\begin{itemize}
    \item L'origine razziale o etnica
    \item Le opinioni politiche
    \item Le convinzioni religiose o filosofiche
    \item L'appartenenza sindacale
    \item Dati genetici
    \item Dati biometrici
    \item Dati relativi alla salute di una persona
    \item Dati relativi alla vita sessuale o all'orientamento sessuale di una persona

\end{itemize}

Questi dati possono essere trattati nelle seguenti condizioni:
\begin{enumerate}
    \item L'interessato ha prestato il proprio consenso \textit{esplicito} al trattamento di tali dati personali per una o più finalità specifiche
    \item Il trattamento è necessario per assolvere gli obblighi ed esercitare i diritti specifici del titolare del trattamento o dell'interessato in materia di diritto del lavoro, e della sicurezza sociale e protezione sociale, in accordo con il diritto autorizzato dagli stati membri o dall'Unione Europea.
    \item Il trattamento è necessario per tutelare l'interesse vitale dell'interessato o di un'altra persona fisica nel caso in cui l'interessato non abbia più capacità fisica o giuridica di prestare il proprio consenso.
    \item Il trattamento viene effettuato da una fondazione, o associazione senza scopo di lucro, che persegue finalità \textit{politiche}, \textit{religiose}, \textit{filosofiche} o \textit{sindacali}, a condizione che ciò avvenga per i membri, ex membri o persone con i contatti
    \item Il trattamento riguarda dati personali resi manifestamente pubblici dall'interessato
    \item Il trattamento è necessario per accertare, esercitare o difendere un diritto in sede giudiziaria o nel momento in cui le autorità giurisdizionali esercitano le loro funzioni giurisdizionali
    \item Il trattamento è necessario per motivi di interesse pubblico rilevante sulla base del diritto europeo o nazionale
    \item Il trattamento è necessario per finalità di medicina o terapia, o per valutare la capacità lavorativa del dipendente
    \item Il trattamento è necessario per motivi di interesse pubblico della sanità pubblica, come la protezione da gravi minacce per la salute, o garanzia di qualità e sicurezza in ambito sanitario
    \item Il trattamento è necessario a fini di archiviazione, ricerca scientifica o a fini statistici, in base al diritto UE o nazionale.
\end{enumerate}
Gli stati membri possono introdurre ulteriori condizioni, anche eventuali limitazioni aggiuntivi riguardo dati genetici, biometrici, o relativi alla salute.
\subsection{Se qualcuno intende trattare i Suoi dati personali, a quali adempimenti è tenuto per rispettare la normativa?}

Vedi domanda 12) (principi applicabili al trattamento dei dati personali)

\subsection{Per quanto tempo posso conservare dei dati personali?}

Secondo il principio di limitazione della conservazione, i dati personali vanno conservati in un arco di tempo non superiore alle finalità per le quali sono trattati.
\newline
Soltanto nei casi in cui vengano conservati per scopi:
\begin{itemize}
    \item di archiviazione
    \item di ricerca scientifica
    \item di ricerca storica
    \item statistici
\end{itemize}
possono essere conservati più a lungo.

\subsection{Che cos’è l’informativa? Chi la deve fornire? Quale contenuto ha (esempi)? Ha un contenuto
obbligatorio minimo? Quando deve essere fornita? Può non essere fornita?}

L'informativa deve informare il soggetto da cui verranno raccolti i dati personali:
\begin{itemize}
    \item dell'identità e i dati di contratto del titolare del trattamento
    \item i dati di contratto del responsabile della protezione dei dati (DPO) se è stato nominato
    \item le finalità del trattamento e la base giuridica del trattamento
    \item eventuali destinatari o categorie di destinatari dei dati personali
    \item l'intenzione del titolare del trattamento di trasferire dati personali a un paese terzo o a un'organizzazione internazionale
    \item il periodo di conservazione dei dati personali o, se non è possibile, i criteri utilizzati per determinare tale periodo
    \item diritti dell'interessato (come accesso ai dati, rettifica, cancellazione, limitazione, opposizione, portabilità
    \item diritto di revocare il consenso in qualsiasi momento senza pregiudicare la liceità del trattamento basata sul consenso prestato prima della revoca
    \item diritto di proporre reclamo a un'autorità di controllo
    \item se la comunicazione di dati personali risulta essere un obbligo legale o contrattuale oppure un requisito necessario per la conclusione di un contratto
    \item l'esistenza di un processo decisionale automatizzato, compresa la profilazione
\end{itemize}
L'informativa deve essere fornita dal titolare del trattamento, e possiede un contenuto obbligatorio minimo.
Deve essere fornita in un momento precedente alla raccolta ed è obbligatoria.

\subsection{Quali sono i diritti dell’interessato?}
I diritto dell'interessato sono:
\begin{itemize}
    \item Diritto di accesso
    \item Diritto di rettifica
    \item Diritto di integrazione
    \item Diritto di cancellazione
    \item Diritto di limitazione del trattamento
    \item Diritto alla portabilità dei dati
    \item Diritto di opposizione
\end{itemize}

\subsection{Se trovate un'informativa di questo tipo: "I Suoi dati saranno trattati nel rispetto del Regolamento
2016/679 per le finalità ivi previste" che considerazioni fate?}

L'informativa con il regolamento sopra citato rispetta la GDPR, e sono previste le condizioni riportate alla domanda 17)

\subsection{Il consenso. Che cos’è? Deve sempre essere richiesto per poter trattare dati personali a norma
di legge? Quali caratteristiche deve avere? Se l'interessato tace, vale come consenso?}

Il consenso è qualsiasi manifestazione di volontà libera, specifica, informata e inequivocabile dell'interessato (art 4 GDPR).
\newline
La richiesta di consenso è obbligatoria per il trattamento dei dati personali.
Il consenso deve essere:
\begin{itemize}
    \item chiaramente distinguibile dalle altre materie
    \item comprensibile
    \item facilmente accessibile
    \item presentato con un linguaggio chiaro e semplice
\end{itemize}
Il silenzio non è considerato come consenso.
\newline
L'articolo esatto che esprime il concetto di consenso è il seguente:
\begin{mdframed}
    [backgroundcolor=blue!20]Il consenso dovrebbe essere espresso mediante un atto positivo inequivocabile con il quale l’interessato manifesta l’intenzione libera, specifica, informata e inequivocabile di accettare il trattamento dei dati personali che lo riguardano, ad esempio mediante dichiarazione scritta, anche attraverso mezzi elettronici, o orale. Ciò potrebbe comprendere la selezione di un’apposita casella in un sito web, la scelta di impostazioni tecniche per servizi della società dell’informazione o qualsiasi altra dichiarazione o qualsiasi altro comportamento che indichi chiaramente in tale contesto che l’interessato accetta il trattamento proposto. Non dovrebbe pertanto configurare consenso il silenzio, l’inattività o la preselezione di caselle. Il consenso dovrebbe applicarsi a tutte le attività di trattamento svolte per la stessa o le stesse finalità. Qualora il trattamento abbia più finalità, il consenso dovrebbe essere prestato per tutte queste. Se il consenso dell’interessato è richiesto attraverso mezzi elettronici, la richiesta deve essere chiara, concisa e non interferire immotivatamente con il servizio per il quale il consenso è espresso.
\end{mdframed}
L'interessato ha inoltre il diritto di revocare il proprio consenso in qualsiasi momento.
\subsection{Vi sono casi in cui non occorre il consenso? Quali?}
Il consenso non viene applicato nel caso in cui il trattamento di dati personali è effettuato da:

Il consenso non viene applicato nel caso in cui il trattamento di dati personali è effettuato da:
\begin{itemize}
    \item Persone fisiche per l'esercizio di attività esclusivamente personali o domestiche
    \item Autorità competenti, ai fini di prevenzioni, indagine o accertamento o perseguimento di reati, o salvaguardia della salute pubblica
\end{itemize}

\subsection{Obblighi e responsabilità del Titolare: misure tecniche e organizzative}

Il titolare deve attuare misure tecniche e organizzative adeguate per garantire, ed essere in grado di dimostrare, che il trattamento è effettuato in modo conforme al Regolamento.
Deve tenere conto dei seguenti aspetti:
\begin{itemize}
    \item natura
    \item ambito di applicazione
    \item contesto
    \item finalità del trattamento
    \item rischi aventi probabilità e gravità diverse per i diritti e le libertà delle persone fisiche.
\end{itemize}

\subsection{Privacy fin dalla progettazione (by design) e per impostazione predefinita (by default)}

\textbf{Privacy by design}:
\newline
Il titolare del trattamento deve mettere in atto misure tecniche e organizzative adeguate, come la pseudonimizzazione, con lo scopo di:
\begin{itemize}
    \item attuare in modo efficace i principi di protezione dei dati, come la minimizzazione
    \item soddisfare i requisiti del regolamento e tutelare i diritti degli interessati
\end{itemize}

Le misure vanno messe in atto:
\begin{itemize}
    \item nel momento in cui vengono determinati i mzzi del trattamento
    \item durante il trattamento stesso
\end{itemize}
\textbf{Privacy by default:}
\newline
Il titolare del trattamento deve garantire che siano messe in atto SOLO misure tecniche e organizzative adeguate per trattare soltanto i dati personali necessari per ciascuna specifica finalità del trattamento
\newline
Questo obbligo vale per:
\begin{itemize}
    \item la quantità dei dati personali raccolti
    \item la portata del trattamento
    \item il periodo di conservazione
    \item l'accessibilità
\end{itemize}
L'impostazione di default prevede che non siano resi accessibili dati personali a un numero indefinito di persone fisiche senza l'intervento della persona fisica.

\subsection{Responsabile del trattamento: spiegare il ruolo e le responsabilità.}

Il responsabile del trattamento fornisce garanzie sufficenti per mettere in atto misure tecniche e organizzative adeguate.
Tratta i dati soltanto su istruzione documentata del titolare del trattamento.
\newline
Può nominare un sotto responsabile solo con autorizzazione scritta del titolare.
\newline
Per vincolare il responsabile al titolare occorre stipulare un contratto che preveda:
\begin{itemize}
    \item la materia disciplinata
    \item la durata del trattamento
    \item la natura del trattamento
    \item la finalità del trattamento
    \item il tipo di dati personali
    \item le categorie di interessati
    \item gli obblighi e i diritti del titolare del trattamento.
\end{itemize}

\subsection{Registri delle attività di trattamento del Titolare e del Responsabile: quando vanno tenuti?
Cosa devono contenere? Perché sono importanti?}

I registri delle attività di trattamento del titolare e del responsabile vanno tenuti da:
\begin{itemize}
    \item Imprese e aziende con più di 250 dipendenti
    \item Imprese o organizzazioni con meno di 250 dipendenti se:
    \begin{itemize}
        \item Il trattamento può presentare un rischi per i diritti e le libertà dell'interessato
        \item Il trattamento non sia occasionali
        \item Include il trattamento di categorie particolari di dati
    \end{itemize}
\end{itemize}

Vanno tenuti in forma scritta, anche in formato elettronico.
Devono contenere (nel registro del titolare):
\begin{itemize}
    \item nome e dati di contratto del titolare del trattamento e del responsabile della protezione dei dati
    \item finalità del trattamento
    \item descrizione delle categorie di interessati e delle categorie di dati personali
    \item categorie di destinatari
    \item trasferimenti di dati personali verso un paese terzo o un'organizzazione internazionale
    \item dove possibile, i termini previsti per la cancellazione delle diverse categorie di dati
    \item dove possibile, una descrizione generale delle misure di sicurezza tecniche e organizzative
\end{itemize}
Il registro del responsabile contiene:
\begin{itemize}
    \item nome e dati di contratto del responsabile o dei responsabili del trattamento, di ogni titolare per conto del quale agisce il responsabile del trattamento
    \item categorie dei trattamenti effettuati per conto di ogni titolare
    \item I trasferimenti di dati personali verso un paese terzo o organizzazione internazionale
    \item identificazione del paese terzo o organizzazione internazionale
    \item descrizione delle misure di sicurezza tecniche e organizzative
\end{itemize}

\subsection{Sicurezza del trattamento – misure di sicurezza}

Il titolare del trattanento e il responsabile del trattamento devono mettere in atto misure tecniche e organizzative adeguate, per garantire un livello di sicurezza adeguato al rischio.
Occorre tenere conto di:
\begin{itemize}
    \item stato dell'arte
    \item costi di attuazione
    \item natura
    \item ogetto
    \item contesto
    \item finalità del trattamento
    \item rischio di varie probabilità e gravità per i diritti e le libertà delle persone fisiche
\end{itemize}
Per le misure di sicurezza occorre tenere conto dei rischi presentato dal trattamento, come ad esempio:
\begin{itemize}
    \item distruzione
    \item perdita
    \item modifica
    \item divulgazione non autorizzata
    \item accesso accidentale o illegale ai dati
\end{itemize}
Esempio di misure di sicurezza sono:
\begin{itemize}
    \item pseudonimizzazione e cifratura dei dati personali
    \item capacità di assicurare la riservatezza, l'integrità, la disponibilità, la resilienza dei sistemi e dei servizi di trattamento
    \item capacità di ripristinare subito la disponilità e l'accesso dei dati personali in caso di incidente fisico o tecnico.
    \item procedura per verificare, testare l'efficacia delle misure tecniche e organizzative
\end{itemize}

\subsection{Notifica violazioni dei dati (data breach): in che cosa consiste? Cosa si deve fare?}

In caso di violazione dei dati personali bisogna notificare il garante della privacy entro 72 ore dal momento in cui si è venuti a conoscenza del fatto, a meno che sia improbabile che la violazione dei dati personali costituisca un rischio per i diritti e per le libertà delle persone fisiche.
\newline
Appena viene a sapere della violazione, il responsabile deve informare il titolare del tratttamento

\subsection{Comunicazione violazioni dei dati (data breach): in che cosa consiste? Cosa si deve fare?}

Quando la violazione dei dati personali presenta un rischio elevato per i diritti e le libertà delle persone fisiche il titolare deve comunicare la violazione all'interessato, senza alcun ritardo, tranne se:
\begin{itemize}
    \item Il titolare ha adottato misure sulle teecniche e organizzative per proteggere
    \item Il titolare ha successivamente adottato misure per scongiurare un rischio elevato
    \item Si procede con una comunicazione pubblica o a una misura simile, tramite la quale gli interessati sono informati con analoga efficacia.
\end{itemize}

\subsection{Valutazione d’impatto sulla protezione dei dati}

Nel caso in cui vengono utilizzate nuove tecnologie, dato che essere possono presentare un rischio elevato per i diritti e le libertà delle persone fisica, il titolare deve effettuare una valutazione di impatti dei trattamenti previsti sulla protezione dei dati personali.

Viene richiesta quando:
\begin{itemize}
    \item occorre una valutazione sistematica e globale basata su un trattamento automatizzato di aspetti personali relativi a persone fisiche.
    In particolare è necessaria quando si fondano decisioni sulla profilazione e che hanno effetti giuridici o incidono sulle persone fisiche.
    \item il trattamento dei dati personali è su larga scala
    \item si ha una zona accessibile al pubblico
\end{itemize}

Il contenuto della valutazione di impatto comprende:
\begin{itemize}
    \item una descrizione sistematica dei trattamenti previsti e delle finalità del trattamento
    \item valutazione della necessità e proporzionalità dei trattamenti in relazione alle finalità
    \item valutazione dei rischi per i diritti e le libertà degli interessati
    \item misure per affrontare i rischi, come:
    \begin{itemize}
        \item garanzie
        \item misure di sicurezza
        \item meccanismi per garantire la protezione dei dati personali e dimostrare la conformità al regolamento
    \end{itemize}
\end{itemize}

\subsection{Responsabile della Protezione dei Dati (DPO): quando deve essere nominato, ruolo, compiti}

Il responsabile della protezione dei dati (DPO) ha lo scopo di supportare l'applicazione del regolamento.
Deve essere esperto sulla normativa e prassi della protezione dei dati, e può essere un dipendente del titolare o del responsabile del trattamento o un soggetto esterno con cui stipula un contratto di servizi.

Non deve essere in conflitto di interesse, né ricevere istruzioni dal titolare o dal responsabile del trattamento.
Un gruppo imprenditoriale può nominare un unico DPO, a patto che sia facilmente raggiungibile da ciascuno stabilimento.

Deve essere nominato quando:
\begin{itemize}
    \item il trattamento viene effettuato da un'autorità pubblica o da un organismo pubblico
    \item le attività principali del titolare del trattamento o del responsabile consistono in trattamenti che richiedono il monitoraggio regolare e sistematico degli interessati su larga scala
    \item Le attività principali consistono del trattamento di dati su larga scala o categorie di dati personali.
\end{itemize}
I compiti minimi del DPO sono:
\begin{itemize}
    \item informare e fornire consulenza al titolare, o al responsabile e ai dipendenti che eseguono il trattamento sugli obblighi del regolamento sulla protezione dei dati
    \item Sorvegliare l'osservanza del regolamento, di altre norme UE sulla protezione dei dati, e delle politiche dei titolare sulla protezione dei dati personali, come l'attribuzione delle responsabilità, la sensibilizzazione e la formazione del personale
    \item Fornire un parere in merito alla valutazione d'impatto sulla protezione dei dati, sorvegliandone lo svolgimento
    \item cooperare con il garante della privacy
    \item fungere da punto di contatto per l'autorità di controllo su questioni connesse al trattamento.
\end{itemize}

\subsection{Trasferimenti di dati extra UE: a quali condizioni sono consentiti?}

I trasferimenti di dati extra in UE può avvenire soltanto se il titolare e il responsabile del trattamento rispettano le condizioni del regolamento.
Le condizioni sono:
\begin{itemize}
    \item decisione di adeguatezza: la commissione europea stabilisce che determinati paesi garantiscono un livello di protezione adeguato, e che non sono necessarie autorizzazioni specifiche
    \item garanzie adeguate: manca una decisione di adeguatezza, ma il titolare o il responsabile forniscono garanzie adeguate
\end{itemize}

In mancanza delle condizioni precedenti i dati possono essere trasferiti alle seguenti condizioni:
\begin{itemize}
    \item consenso dell'interessato, dopo che è stato informato sui rischi dei trasferimenti, dovuti alla mancanza delle decisioni di adeguatezza e delle garanzie adeguate
    \item Trasferimento necessario all'esecuzione di un contratto concluso tra l'interessato e il titolare del trattamento
    \item Trasferimento necessario per la conclusione o l'esecuzione di un contratto stipulato tra titolare del trattamento e un'altra persona fisica o giuridica
    \item Trasferimento necessario per importanti motivi di interesse pubblico
    \item Trasferunebti necessariio per accertare, esercitare o difendere un diritto in sede giudiziaria
    \item Trasferimento necessario per tutelare gli interessi vitali dell'interessato o di altre persone, nel caso in cui l'interessato si trovi nell'incapacità fisica o giuridica di presare il proprio consenso.
    \item Trasferimento effettuato a partire da un registro pubblico
\end{itemize}
\subsection{A quali condizioni possono essere inviati SMS, e-mail pubblicitarie?}
Possono essere inviati soltanto se l'interessato ha espresso il consenso esplicito per ottenere SMS o email pubblicitarie in una clausola apposta prevista dal contratto.

\subsection{Qual è l'attuale disciplina per le telefonate pubblicitarie? Cosa può fare per tutelarsi? Il Registro
pubblico delle opposizioni.}

Per tutelarsi esiste un registro apposito, noto come registro pubblico delle opposizioni, che è un servizio gratuito offerto dallo stato che consente ai cittadini di essere tutelati dalle chiamate indesiderate, da call center che propongono offerte telefoniche, cosmetici, vino ecc.

Il registro è una banca dati che contiene un elenco di nominati e codici utenza degli abbonati, e solo il garante per la protezione dei dati personali e l'autorità giudiziaria possono averne accesso.
La registrazione risulta gratuita. Se i callcenter continuano a chiamare utenti registrati nel registro pubblico delle oppposizioni rischiano multe da 30 mila euro a 180 mila euro.

\subsection{Se un’impresa ha un sito web (es. di e-commerce), quali sono i principali problemi di privacy
che deve affrontare?}

I problemi che possono presentarsi sono i seguenti:
\begin{itemize}
    \item gli utenti provengono da tutto il mondo, pertanto il trattamento dei dati personali non dipende soltanto dalla GDPR, ma anche da norme interne allo stato dell'utente coinvolto. Occorre inoltre tradurre la privacy policy.
    \item può essere necessario trattare dati particolari come articoli di carattere religioso e politico.
    Occorre quindi adottare specifiche misure di sicurezza per tutelare questi dati particolari, anche nominando, se necessario, un DPO.
    \item occorre adottare sistemi di profilazione per studiare le preferenze e le scelte di acquisto dei propri clienti per aumentare le proprie capacità di vendita e per avere offerte mirate. Anche in questo caso è prevista una clausola apposita nel trattamento dei dati personali.
    \item Per inviare newsletter o comunicazioni di prodotti o servizi ai clienti occorre l'approvazione esplicita di una clausola apposita da parte del cliente, specie se le newsletter provengono da enti terzi.

\end{itemize}