\newpage
\section{Proprietà industriale}

\subsection{Cosa tutela la proprietà industriale?}
La proprietà industriale comprende:
\begin{itemize}
    \item marchi e altri segni distintivi
    \item indicazioni geografiche
    \item denominazioni di origine
    \item disegni e modelli
    \item invenzioni
    \item modelli di utilità
    \item topografie dei prodotti a semiconduttori
    \item segreti commerciali
    \item nuove varietà vegetali
\end{itemize}

\subsection{Quando sorgono e come si acquisiscono i diritti di proprietà industriale?}
I diritti di proprietà industriale si acquistano mediante \textbf{brevettazione, registrazione} e negli altri
modi previsti dal Codice della proprietà industriale (CPI). \newline
Sono oggetto di brevettazione le invenzioni, modelli di utilità, nuove varietà vegetali. \newline
Sono oggetto di registrazione: marchi, disegni, modelli, topografie dei prodotti a semiconduttori.\newline
Le facoltà esclusive attribuite dal CPI al titolare di un diritto di proprietà industriale si esauriscono una volta
che i prodotti protetti da un diritto di proprietà industriale siano stati messi in commercio dal titolare o con
il suo consenso nel territorio dello Stato o nel territorio di uno Stato membro della Comunità Europea o dello spazio
economico europeo. Esempio di esaurimento delle facoltà esclusive attribuite dal CPI: \newline

il titolare “A” di un marchio che contraddistingue un capo di abbigliamento immette in commercio il capo medesimo, ponendolo in vendita; il capo viene acquistato dal soggetto “B”, che successivamente lo rivende al soggetto “C”. “C” è libero di utilizzare il capo come meglio crede, utilizzandolo o rivendendolo a sua discrezione senza alcuna necessità di licenza od altra autorizzazione da parte del titolare “A”, essendosi le facoltà esclusive di quest’ultimo esaurite con la prima immissione in commercio del bene.

\subsubsection{I marchi oggetto di registrazione}
Possono costituire oggetto di registrazione come marchio d'impresa tutti i segni, in particolare le parole, compresi i nomi di persone,
i disegni, le lettere, le cifre, i suoni, la forma del prodotto o della confezione di esso, le combinazioni o le tonalità cromatiche, purchè siano atti a:
\begin{itemize}
    \item distinguere i prodotti o i servizi di un'impresa da quelli di altre imprese e
    \item essere rappresentati nel registro in modo tale da consentire alle autorità competenti ed al pubblico di determinare con chiarezza e precisione l'oggetto della
    protezione conferita al titolare
\end{itemize}
Quando si registra un marchio non lo si può fare per qualsiasi attività ma occorre identificare i prodotti/servizi per i quali si deposita il marchio, ossia
le cosiddette \emph{Classi}. Un marchio si può registrare per una o più classi.

\subsection{Quali segni non possono essere registrati come marchi?}
Non possono essere registrati come marchio d'impresa i segni che alla data del deposito della domanda:
\begin{itemize}
    \item siano identici o simili a un segno già noto come marchio o segno distintivo di prodotti o servizi
    fabbricati, messi in commercio o prestati da altri per prodotti o servizi identici o affini, se a causa dell'identità o somiglianza tra i segni
    e dell'identità o affinità fra i prodotti o i servizi possa determinarsi un rischio di confusione o associazione fra i due segni.\newline
    L'uso precedente del segno, quando non importi notorietà di esso, o importi notorietà puramente locale, non toglie la novità (cioè si può registrare il marchio)
    MA il terzo preutente (inteso come l'altro preutente) ha diritto di continuare nell'uso del marchio, anche ai fini della pubblicità, nei limiti
    della diffusione locale, nonostante la registrazione del marchio stesso. L'uso precedente del segno da parte
    del richiedente o del suo dante causa non è di ostacolo alla registrazione
    \item siano identici o simili a un segno già noto come ditta, denominazione o ragione sociale, insegna e nome a dominio, o altro segno distintivo adottato
    da altri, se a causa dell'identità o somiglianza fra i segni e dell'identità o affinità fra l'attività d'impresa da questi esercitata e i prodotti o servizi per i quali
    il marchio è registrato possa sorgere un rischio di confusione per il pubblico o un rischio di associazione fra i due segni.\newline
    L'uso precedente del segno, quando non importi notorietà di esso, o importi notorietà puramente locale, non toglie la novità. L'uso precedente del segno da parte del richiedente
    o del suo dante causa non è di ostacolo alla registrazione
    \item siano identici ad un marchio già da altri registrato nello Stato o con efficacia nello Stato per prodotti o servizi identici
    \item siano identici o simili ad un marchio già da altri registrato nello Stato o con efficacia nello Stato per prodotti o servizi identici
    o affini, se a causa dell'identità o somiglianza fra i segni e dell'identità o affinità fra i prodotti o i servizi possa determinarsi un rischio di confusione per il pubblico,
    che può consistere anche in un rischio di associazione fra i due segni
    \item siano identici o simili ad un marchio già da altri registrato nello Stato o con efficacia nello Stato per prodotti o servizi anche non affini,
    quando il marchio anteriore goda nella Comunità, se comunitario, o nello Stato, di rinomanza e quando l'uso di quello successivo senza giusto motivo trarrebbe indebitamente vantaggio
    dal carattere distintivo o dalla rinomanza del segno anteriore o recherebbe pregiudizio agli stessi
    \item siano identici o simili ad un marchio già notoriamente conosciuto ai sensi dell'articolo 6-bis della Convenzione di Parigi per la protezione della proprietà industriale, per prodotti
    o servizi anche non affini, quando ricorrono le condizioni di cui al punto precedente

\end{itemize}

\subsection{Quali segni non possono essere oggetto di tutela per mancanza di capacità distintiva?}
Non possono costituire oggetto di registrazione come marchio d'impresa i segni privi di carattere distintivo, in particolare
quelli che sono esclusivamente divenuti di uso comune nel linguaggio comune o negli usi costanti del commercio, e quelli
costituiti esclusivamente dalle denominazioni generiche di prodotti o servizi o da indicazioni descrittive che ad essi si
riferiscono, come i segni che in commercio possono servire a designare la specie, la qualità, la quantità, la destinazione,
il valore, la provenienza geografica ovvero l'epoca di fabbricazione del prodotto o della prestazione del servizio o altre caratteristiche
del prodotto o servizio.\newline
Un marchio può decadere se, per il fatto dell'attività o dell'inattività del suo titolare, sia divenuto nel commercio denominazione
generica del prodotto o servizio o abbia comunque perduto la sua capacità distintiva

\subsection{Chi può registrare un marchio?}
Può registrare un marchio d'impresa chi lo utilizzi o si proponga di utilizzarlo, nella fabbricazione o commercio di prodotti
o nella prestazione di servizi della propria impresa o di imprese di cui abbia il controllo o che ne facciano uso con il suo consenso.\newline
Anche le amministrazioni dello Stato, delle regioni, delle province e dei comuni possono registrare marchi, anche aventi ad oggetto elementi grafici distintivi
tratti dal patrimonio culturale, storico, architettonico o ambientale del relativo territorio. \newline

\subsection{Quali sono i diritti del titolare di un marchio?}
Il primo diritto del titolare di un marchio è quello di fare uso esclusivo del marchio. Esso infatti può vietare a terzi, salvo
proprio consenso, di usare nell'attività economica:
\begin{itemize}
    \item un segno identico al marchio per prodotti o servizi identici a quelli per cui esso è stato registrato
    \item un segno identico o simile al marchio, per prodotti o servizi identici o affini, se dall'identità o somiglianza
    fra i segni e dall'identità o affinità fra i prodotti o servizi, possa derivare un rischio di confusione per il pubblico,
    o un rischio di associazione fra i due segni
    \item un segno identico o simile al marchio registrato per prodotti o servizi anche non affini, se il marchio registrato
    goda nello stato di rinomanza e se l'uso del segno senza giusto motivo consente di trarre indebitamente vantaggio dal carattere
    distintivo o dalla rinomanza del marchio o reca pregiudizio agli stessi
\end{itemize}

\subsection{Quali sono le limitazioni al diritto di marchio?}
Il titolare dei diritti di marchio registrato non può vietare ai terzi l'uso nell'attività economica, purchè l'uso sia conforme ai principi
della correttezza professionale del loro nome o indirizzo, delle caratteristiche (specie, qualità, provenienze ecc.) del prodotto o del servizio e del marchio
d'impresa se esso è necessario per indicare la destinazione di un prodotto o servizio, in particolare come accessori o pezzi di ricambio. \newline
Non si può usare il marchio in modo contrario alla legge, nè in modo da ingenerare un rischio di confusione sul mercato con altri segni
conosciuti come distintivi di imprese, prodotti o altri servizi altrui, o da indurre comunque in inganno il pubblico, in particolare circa la natura,
qualità o provenienza dei prodotti o servizi, a causa del modo e del contesto in cui viene utilizzato, o da ledere un altrui diritto di autore,
di proprietà industriale, o altro esclusivo di terzi. \newline
È vietato a chiunque di fare uso di un marchio registrato dopo che la relativa registrazione è stata dichiarata nulla,
quando la causa di nullità comporta l'illeicità dell'uso del marchio.

\subsection{Il trasferimento del marchio}
Il marchio può essere trasferito per la totalità o per una parte dei prodotti o servizi per i quali è stato registrato. \newline
Il marchio può essere oggetto di licenza anche non esclusiva per la totalità o per parte dei prodotti o dei servizi
per i quali è stato registrato e per la totalità o per parte del territorio dello Stato, a condizione che in caso di licenza non esclusiva,
il licenziatario si obblighi espressamente ad usare il marchio per contraddistinguere prodotti o servizi eguali a quelli corrispondenti
messi in commercio o prestati nel territorio dello Stato con lo stesso marchio dal titolare o da altri licenziatari. \newline
Il titolare del marchio d'impresa può far valere il diritto all'uso esclusivo del marchio stesso contro il licenziatario che violi le disposizioni
del contratto di licenza relativamente alla durata, al modo di utilizzazione del marchio, alla natura dei prodotti o servizi per i quali la licenza
è concessa, al territorio in cui il marchio può essere usato o alla qualità dei prodotti fabbricati e dei servizi prestati dal
licenziatario

\subsection{Il brevetto per invenzione industriale}
Possono costituire oggetto di brevetto per invenzione le invenzioni, di ogni settore della tecnica, che sono nuove e che implicano
un'attività inventiva e sono atte ad avere un'applicazione industriale.\newline
\emph{Non} sono considerate invenzioni le scoperte, le teorie scientifiche e i metodi matematici, i piani
e i metodi per attività intellettuali/ludiche/commerciali e i programmi per elaboratore. Anche la presentazione di informazioni non è
considerata invenzione.\newline
Questi elementi sono intesi in quanto tali.

\subsection{Il requisito di novità delle invenzioni}
Un’invenzione è considerata nuova se non è già compresa nello stato della tecnica; ove per stato della tecnica si intende tutto ciò che è stato reso accessibile al pubblico, in Italia o all’estero, prima della data del deposito della domanda di brevetto mediante descrizione scritta od orale, una utilizzazione o un qualsiasi altro mezzo (Rif. Art 46 CPI). Ad esempio, se un’invenzione identica a quella oggetto della domanda di brevetto è già stata realizzata da un terzo, ma mai divulgata, sarà possibile procedere ugualmente al deposito della domanda; se, invece, quest’ultimo l’ha già diffusa in qualunque modo in Italia o all’estero, l’altrui invenzione non potrà più essere considerata nuova. Anche la pubblicazione dell’invenzione in un giornale scientifico, la relativa presentazione in una conferenza, l’utilizzo in ambito commerciale, l’esposizione in un catalogo costituiscono atti in grado di annullare la novità dell’invenzione.
\newline Una divulgazione dell'invenzione non è rpesa in considerazione se si è verificata nei 6 mesi che precedono
la data di deposito della domanda di un brevetto e se risulta direttamente o indirettamente da un abuso evidente ai danni del richiedente
o del suo dante causa.

\subsection{Il concetto di attività inventiva e industrialità}
Un'invenzione è considerata come implicante un'attività inventiva se, per una persona esperta del ramo, essa non risulta in modo evidente
dallo stato della tecnica.\newline
Un'invenzione è considerata atta ad avere un'applicazione industriale se il suo oggetto può essere fabbricato o utilizzato in qualsiasi genere di
industria, compresa quella agricola.

\subsection{Il concetto di priorità}
Chi deposita una domanda di brevetto in uno Stato o per uno Stato facente parte di una Convenzione internazionale ratificata dall'Italia che riconosce
il diritto di priorità ha un periodo di 12 mesi in cui valutare l'ambito territoriale in cui richiedere la tutela, senza dover depositare contemporaneamente
domanda in tutti i paesi di potenziale interesse (In dottrina viene chiamato "periodo di riflessione").\newline
Il titolare gode di un diritto di priorità dalla data di deposito; questo influisce sul momento di determinazione della novità anche negli Stati in cui il deposito
sarà fatto successivamente.\newline
La novità delle domande che rivendicano la priorità non potrà quindi essere compromessa da "anteriorità opponibili" (cioè domande fatte da terzi tra il primo
deposito ed eventuali depositi successivi consentiti dalla priorità)

\subsection{Come si fa domanda di brevetto}
La domanda deve essere redatta su apposito modulo scaricabile dal sito www.uibm.gov.it (Ufficio Italiano Brevetti e Marchi). Il deposito può essere effettuato online, presso una
qualsiasi Camera di Commercio, oppure inviata direttamente all'Ufficio Italiano Brevetti e Marchi mediante raccomandata A/R.\newline
Nella domanda bisogna allegare:
\begin{itemize}
    \item un riassunto con disegno principale
    \item un riassunto, senza disegni, la descrizione vera e proprio e le rivendicazioni: \newline
    nelle rivendicazioni è indicato, specificamente, ciò che debba formare oggetto del brevetto. Esse determinano i limiti
    della protezione prevista dal brevetto.
    \item il disegno, o i disegni, dell'invenzione
    \item la versione in lingua inglese delle rivendicazioni per le sole invenzioni per le quali non si
    rivendichi una priorità interna o estera
    \item la versione inglese del riassunto e della descrizione (opzionale)
    \item la ricevuta del pagamento dei diritti
    \item la designazione dell'inventore
    \item il documento di priorità
\end{itemize}

\subsection{Quali sono gli effetti della brevettazione?}
I diritti esclusivi sulle invenzioni industriali in base al Codice sono conferiti con la concessione del brevetto. \newline
Gli effetti del brevetto decorrono dalla data in cui la domanda con la descrizione, le rivendicazioni e gli eventuali disegni è resa accessibile al pubblico.\newline
Trascorsi 18 mesi dalla data di deposito della domanda o dalla data di priorità, o dopo 90 giorni dalla data di deposito della domanda se il
richiedente ha dichiarato nella domanda stessa di volerla rendere immediatamente accessibile al pubblico, l'Ufficio italiano brevetti e marchi mette a disposizione del
pubblico la domanda con gli allegati. \newline
Nei confronti delle persone alle quali la domanda è stata notificata a cura del richiedente, gli effetti del brevetto per invenzione
industriale decorrono dalla data di tale notifica.

\subsection{Il brevetto europeo}
Il brevetto europeo rilasciato per l'Italia conferisce gli stessi diritti ed è sottoposto allo stesso regime dei brevetti italiani a decorrere alla data
in cui è pubblicata nel Bollettino europeo dei brevetti la menzione della conessione del brevetto.\newline
Qualora il brevetto sia soggetto a procedura di opposizione ovvero di limitazione, l'ambito della protezione stabilito con la concessione o con la decisione
di mantenimento in forma modificata o con la decisione di limitazione è confermato a decorrere dalla data in cui è pubblicata la menzione della decisione
concernente l'opposizione o la limitazione. \newline
Il titolare deve fornire all'Ufficio italiano brevetti e marchi una traduzione in lingua italiana del testo del brevetto concesso dall'Ufficio europeo
nonchè del testo del brevetto mentenuto in forma modificata a seguito della procedura di opposizione o limitato a seguito della procedura di limitazione.\newline
La traduzione, dichiarata perfettamente conforme al testo originale dal titolare del brevetto ovvero dal suo mandatario, deve essere depositata entro tre mesi. \newline
Se non è fornita la traduzione e se non è depositata nei termini, il brevetto europeo è consideato, fin dall'origine,
senza effetto in Italia.\newline
La traduzione in lingua italiana degli atti relativi alla domanda depositata o al brevetto europeo concesso è considerata facente fede nel territorio dello
Stato, qualore conferisa una protezione meno estesa di quella conferita dal testo originario. Una traduzione rettificata può essere presentata, in qualsiasi momento, dal
titolare della domanda o del brevetto.

\subsection{Qual è la durata di un brevetto?}
Il brevetto per invenzione industriale dura 20 anni e decorre dalla data di deposito della domanda. \newline
Non può essere rinnovato, nè può esserne prorogata la durata.\newline
La durata è di 10 anni nel caso il brevetto riguardi un modello di utilità, cioè un modello che dia una forma nuova di un
prodotto industriale che dà al prodotto stesso una particolare efficacia o comodità di applicazione o di impiego. \newline
L'oggetto del brevetto deve essere attuato entro 3 anni dalla data di concessione del brevetto e l'attuazione non deve essere sospesa per più
di tre anni consecutivi.\newline
In caso di mancato pagamento dei diritti entro 6 mesi dalla scadenza del diritto annuale dovuto, il brevetto decade.

\subsection{In cosa consistono i dirtti morali e i diritti patrimoniali?}
Il diritto morale consiste nell'essere riconosciuto come autore dell'invenzione; questo può essere fatto valere
dall'inventore e, dopo la sua morte, dal coniuge e dai discendenti fino al secondo grado; in loro mancanza o dopo
la loro morte, dai genitori e dagli altri ascendenti ed in mancanza, o dopo la morte di questi, dai parenti fino al quarto
grado incluso. \newline
I diritti patrimoniali sono i diritti nascenti dalle invenzioni industriali, tranne il diritto di essere riconosciuto come
autore, sono alienabili e trasmissibili.\newline
Il diritto al brevetto per invenzione industriale spetta all'autore dell'invenzione e ai suoi aventi causa (cioè coloro che acquistano il diritto
in quanto è stato loro trasferito da parte del precedente titolare [dante causa]).

\subsection{Le invenzioni dei dipendenti}
Quando l'invenzione industriale è fatta nell'esecuzione o nell'adempimento di un contratto o di un rapporto di lavoro o d'impiego, in cui
l'attivitò inventiva è prevista come oggetto del contratto o del rapporto e a tale scopo retribuita, i diritti derivanti dall'invenzione stessa
appartengono al datore di lavoro, salvo il diritto spettante all'inventore di esserne riconosciuto autore.\newline
Se non è prevista e stabilita una retribuzione, in compenso dell'attività inventiva, e l'invenzione è fatta nell'esecuzione o nell'adempimento
di un contratto o di un rapporto di lavoro o di impiego, accade che i diritti derivanti dall'invenzione appartengono al datore di lavoro,
ma all'inventore, salvo sempre il diritto di essere riconosciuto autore, spetta, qualora il datore di lavoro o i suoi aventi causa ottengano il brevetto
o utilizzino l'invenzione in regime di segretezza industriale, un equo premio per la determinazione del quale si terrà conto dell'importanza
dell'invenzione, delle mansioni svolte e della retribuzione percepita dall'inventore, nonchè del contributo che questi ha ricevuto dall'organizzazione
del datore di lavoro.\newline
Se non ricorrono le condizioni precedenti e l'invenzione industriale rientra nel campo di attività del datore di lavoro, il datore di lavoro
ha il diritto di opzione per l'uso esclusivo o non esclusivo dell'invenzione o l'acquisto del brevetto, nonchè la facoltà di chiedere od acquisire, per
la medesima invenzione, brevetti all'estero. Dovrà pagare all'inventore un canone o un prezzo, da fissars con deduzione di una somma corrispondente agli aiuti
che l'inventore abbia ricevuto dal datore di lavoro per sviluppare l'invenzione. Il datore di lavoro potrà esercitare il diritto di opzione entro 3 mesi
dalla data di ricevimento della comunicazione dell'avvenuto deposito della domanda di brevetto.\newline
\newline
Si ponga attenzione che viene considerata fatta durante l'esecuzione del contratto o del rapporto di lavoro o d'impiego l'invenzione industriale
per la quale sia chiesto il brevetto entro un anno da quando l'inventore ha lasciato l'azienda privata o l'amministrazione pubblica nel cui campo di
attività l'invenzione rientra.

\subsection{Quando/come sorgono i diritti esclusivi su di un'invenzione industriale?}
I diritti esclusivi su di un'invenzione industriali coincidono con i diritti di brevetto e consistono nella facoltà esclusiva di
attuare l'invenzione e di trarne profitto nel territorio dello Stato, entro i limiti ed alle condizioni previste dal presente codice.\newline
In particolare, il brevetto conferisce al titolare i seguenti diritti esclusivi:
\begin{itemize}
    \item se oggetto del brevetto è un prodotto, il diritto di vietare ai terzi, salvo consenso del titolare, di produrre, usare, mettere in commercio,
    vendere o importare a tali fini il prodotto in questione
    \item se oggetto del brevetto è un procedimento, il diritto di vietare ai terzi, salvo consenso del titolare, di applicare il procedimento, nonchè di usare,
    mettere in commercio, vendere o importare a tali fini il prodotto direttamente ottenuto con il procedimento in questione.
\end{itemize}

\subsection{La nullità del brevetto}
Il brevetto è nullo se:
\begin{itemize}
    \item è privo dei requisiti richiesti
    \item rientra in una fattispecie espressamente esclusa dalla brevettabilità
    \item la descrizione non è sufficientemente chiara e completa
    \item l'oggetto si estende oltre il contenuto della domanda iniziale
    \item il titolare non aveva diritto ad ottenerlo
\end{itemize}
Se le cause di nullità colpiscono solo parzialmente il brevetto, la relativa sentenza di nullità parziale comporta una corrispondente
limitazione del brevetto stesso, e nel caso previsto dall'articolo 79, comme 3, stabilisce le nuove rivendicazioni conseguenti alla limitazione.\newline
Il brevetto nullo può produrre gli effetti di un diverso brevetto del quale contenga i requisiti di validità e che sarebbe stato voluto dal
richiedente, qualora questi ne avesse conosciuto la nullità. La domanda di conversione può essere proposta in ogni stato e grado del giudizio.\newline
Il titolare del brevetto convertito, entro 6 mesi dal passaggio in giudicato della sentenza di conversione, presenta domanda di correzione del testo del brevetto.
L'Ufficio, verificata la corrispondenza del testo alla sentenza, lo rende accessibile al pubblico.\newline
Qualora la conversione comporti il prolungamento della durata originaria del brevetto nullo, i licenziatari e
coloro che in vista della prossima scadenza avevano compiuto investimenti seri ed effettivi per utilizzare l'oggetto del brevetto hanno diritto di ottenere licenza obbligatoria
e gratuita non esclusiva per il periodo di maggior durata.\newline
La declaratorio di nullità del brevetto ha effetto retroattivo, ma non pregiudica:
\begin{itemize}
    \item gli atti di esecuzione di sentenze di contraffazione passate in giudicato già compiuti
    \item i contratti aventi ad oggetto l'invenzione conclusi anteriormente al passaggio in giudicato della sentenza
    che ha dichiarato la nullità nella misura in cui siano già stati eseguiti
    \item i pagamenti già effettuati a titolo di equo premio, canone o prezzo.
\end{itemize}
